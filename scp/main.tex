\documentclass[acmsmall,colorlinks, dvipsnames]{acmart}
%% colorlinks: option for package hyperref
%% dvipsnames: option for package xcolor

\usepackage{mystyle}
\usepackage{mymacros}
\usepackage{booktabs}
\usepackage{xtab}
\usepackage{physics}
\usepackage{amsmath}
\usepackage{tikz}
\usepackage{mathdots}
\usepackage{yhmath}
\usepackage{cancel}
\usepackage{color}
\usepackage{siunitx}
\usepackage{array}
\usepackage{multirow}
\usepackage{amssymb}
\usepackage{gensymb}
\usepackage{booktabs}
\usepackage{array}
\usetikzlibrary{fadings}
\usetikzlibrary{patterns}
\usetikzlibrary{shadows.blur}
\usetikzlibrary{shapes}
\usepackage{lineno}
\usepackage[normalem]{ulem}
\usepackage{tablefootnote}
\usepackage{colortbl}
\usepackage{makecell}
\usepackage[table]{xcolor}
\usepackage{tabularx}
\usepackage{threeparttable} %footnote for tables

\usepackage{url}
\def\UrlBreaks{\do\/\do-}
\usepackage{breakurl}
\usepackage{hyperref}
\usepackage{framed}

% Stretch whitespace a little bit to avoid overfull.
\emergencystretch=1em

\setcitestyle{numbers, comma}

%%Este comando eh usado para referenciar código PVS
% \newcommand{\pvscode}[1]{\texttt{#1}} 
\newcommand{\pvscode}[1]{\ensuremath{\mathit{#1}}} 
\newtheorem{axiom}{Axiom}

%\journal{Science of Computer Programming}

%%
%% These commands are for a JOURNAL article.
\acmJournal{TOSEM}

%% The following commands result in compilation errors if given empty arguments.
% \acmVolume{}
% \acmNumber{}
% \acmArticle{}
% \acmMonth{}



%\bibliographystyle{elsarticle-harv}
\bibliographystyle{model2-names}


\begin{document}
 
\selectlanguage{english}

%\begin{frontmatter}

%\linenumbers

\title{Analysis of Command and Control Strategy Through the Interaction of Software Components}

    % Authors
    
%Thiago
%Leopoldo
%Vander
%Sven
%Maxime
%Rohit

\author{Eduardo Lemos Rocha}
\affiliation{%
  \institution{University of Brasília}
  \streetaddress{Campus Universitário Darcy Ribeiro - Edifício CIC/EST, 70910-900}
  \city{Brasilia}
  \country{Brazil}}
\email{dudulr10@gmail.com}

\author{Vander Alves}
\affiliation{%
  \institution{University of Brasília}
  \streetaddress{Campus Universitário Darcy Ribeiro - Edifício CIC/EST, 70910-900}
  \city{Brasilia}
  \country{Brazil}}
\email{valves@unb.br}


%%
%% The abstract is a short summary of the work to be presented in the
%% article.
\begin{abstract}

A group of entities coordinating with each other to accomplish a common objective needs to cope with dynamic events to achieve success. By changing its current organization, the group can avoid unsatisfying results caused be the embedded dynamism related to the scenario. Although the presence of unexpected incidents is recognized in real world scenarios, the current state-of-art does not explore methodologies to increase the capability of handling new circumstances by changing the group's structure. To address this absence, we propose a computational model of a typed-parameterized extension of a channel system and its implementation in the Java Platform. The model is capable of dealing with context changes by providing communication among the group and coordination adjustments when necessary. To evaluate our solution, we conduct a simulation, which exposes the difference between the proposed model and the baseline approach, by a comparison between a series of established metrics.

\end{abstract}

%%
%% The code below is generated by the tool at http://dl.acm.org/ccs.cfm.
%%
\begin{CCSXML}
<ccs2012>
   <concept>
       <concept_id>10011007.10011074.10011092.10011096.10011097</concept_id>
       <concept_desc>Software and its engineering~Software product lines</concept_desc>
       <concept_significance>500</concept_significance>
       </concept>
   <concept>
       <concept_id>10011007.10011074.10011099.10011692</concept_id>
       <concept_desc>Software and its engineering~Formal software verification</concept_desc>
       <concept_significance>500</concept_significance>
       </concept>
 </ccs2012>
\end{CCSXML}

\ccsdesc[500]{Software and its engineering~Computing methodologies~Modeling and simulation~Simulation evaluation}
\ccsdesc[500]{Applied computing~Computers in other domains~Military}

\keywords{Command and Control, C2 Maneuver, Coordination, Channel System, Simulation}

%%
%% This command processes the author and affiliation and title
%% information and builds the first part of the formatted document.
\maketitle

\section{Introduction}
\label{sec:introduction}

\textit{Command and Control (C2)} is about focusing the efforts of a set of entities and resources towards the achievement of some task, objective, or goal~\citep{CC02}. The entities may represent individuals or organizations, and resources involve everything manipulated by the entities, including information exchange. Originally developed in military domain, C2 was based on the idea of a central command concentrating information and power over required elements to accomplish the mission~\citep{CC01}.

In the information age, C2 has developed to absorb the new resources brought by modern technologies in the information exchange process. The result of these developments has been the application of C2 in several domains, such as by Civil Defense during disaster relief operations, and financial operations managing resources to maximize results~\citep{CC03,CC04}. Further C2 applications include sensitive issues as nuclear weapon and research~\citep{C2-EX2}, national mass-vaccination campaigns~\citep{C2-EX1}, and the COVID-19 pandemic scenario, orchestrating different government and research organizations to find a mitigation or solution to the problem and providing responses to the society~\citep{C2-EX3, C2-EX4, C2-EX5}.

The complexity of existing scenarios in different C2 domains stems mostly from the inherent dynamism present in all of them due to changes of circumstance or context. For instance, the sudden lack of entities or the increased risk of the scenario are some of the problems that characterize circumstance changes of C2 applied in the Civil Defense domain. Such scenario changes characterize context dynamism, which can occur in the mission, environment, or entity.

\textit{C2 agility} is the entity's capability of dealing with context changes~\citep{france2014}. To cope with some of these, providing the entities with new resources has been reported to suffice~\citep{france2014}. On the other hand, more complex scenarios caused by new circumstances require changing the collaboration approach among entities to deal with this dynamism in a suitable way. Essentially, C2 agility is a timely and effective response to context changes.

Nevertheless, there is a lack of evidence on how to provide C2 agility. Indeed, state-of-the-art and state-of-the-practice approaches rarely explore context changes. The few that do focus on randomized network-level reconfiguration rather than on entity-level reconfiguration and thus do not deal adequately with the adaptation of the mission accomplishment, which does not characterize the C2 agility.~\citep{france2014,Alberts2017,evaluating,Alberts2011, nato01}. On the other hand, dynamism abounds in real scenarios  and thus are under circumstance changes all the time~\cite{evaluating}. This aspect highlights C2 agility's relevance to deal with context changes providing capability of adaptation. In this paper, we address this problem within the scope of simulated environments, which is often addressed in C2~\cite{CC02} in general and in its application areas such as the military domain~\cite{france2014}.

To provide C2 agility, we present a computational model that coordinates and reconfigures entities to handle context changes. Such model is a typed-parameterized extension of a channel system~\citep{modelcheckingBaier}. This extension, hereafter referred to as CS, defines the roles and responsibilities that are executed by the modelled entities. Each member executing roles is modelled as a Dynamic Software Product Line (DSPL)~\citep{Bencomo2008}. To cope with context changes, members can reconfigure themselves or change their coordination structure thereby achieving C2 agility.

Although modern techniques of Model Checking support a large number of states~\citep{modelcheckingBaier}, concurrent systems like ours may suffer from the possibility of states number explosion. \cite{clarkson2014} showed a prototype of Model Checker to QPTL~\citep{QPTL001} resulted from a transformation of HyperCTL formula. However the checker is impractical for real-world programs and does not provide a timely response in runtime when the system is under a dynamic context. Based on this, we evaluate the proposed C2 computational model performing an agility assessment in a software-based simulated environment.

Even the simulation environment being simpler than real scenarios, it is relevant to a myriad of different domains such as the military~\citep{CC03} and environmental monitoring~\cite{simulation001}. Simulation creates different circumstances to evaluate a solution or product, or to train professionals, reducing the need for resources to create real circumstances. Furthermore, many scenarios occur naturally in virtual environment, such as Network Centric Warfare and telemedicine, making simulations relevant to represent events that occur in virtual world
~\citep{telemedicine01, france2014, CC01}.

In our study, we simulate a set of UAVs applied in a reconnaissance mission. The simulation explores different scenarios with context changes. Such changes occur in the mission due to eventually tasks addition, in the entities due to aleatory damages in onboard  entities’ components and parts, or in the environment due to hazard increasing or weather conditions changes, thus causing impacts on the execution. The results of quantitative and qualitative agility metrics indicate that the entities have a certain level of agility. We also identify challenging situations in achieving agility and discuss related tradeoffs. In summary, this work makes the following contributions:

\begin{itemize}
    \item We present a typed-paramaterized channel system modelling C2 system roles, their interactions, and dealing with context changes (Section~\ref{sec:channelSystem});
    \item We design and implement the proposed  channel system and make it  publicly available\footnote{http://github.com/c2} (Section~\ref{sec:designImpl});
    \item We perform a simulation-based study to empirically evaluate the proposed computational model in providing C2 agility, according to quality and quantity metrics (Section~\ref{sec:evaluation}). 
\end{itemize}

\section{Motivating Example}
\label{sec:motivation}

% Context and Explanation of the example
To provide intuition about the problem, we present an exemplification of a mission execution by entities in a military context. Figure \ref{fig:2TeamExecution} illustrates a reconnaissance mission with some tasks randomly distributed requiring different types of sensors. In this mission, a team comprising four Unmanned Aerial Vehicles (UAVs) is interacting in a network configuration with a central leader, which is responsible only for providing instructions to its subordinates, e.g, distributing tasks among members. These interactions and network topology configure a specific C2 approach, named as \textit{Coordinated}~\citep{france2014}, based on the centralized distribution of information, patterns of interaction, and decision rights. The leader, marked in blue, guides the other team members to complete the tasks, i.e., obtain aerial images of particular points in the field, represented by red crosses.

% Highlighting the problem
The mission involves some natural risks that may lead to change in the conditions of execution. In our example, one member of the team, marked in red, fell due to some environmental change, such as an intense storm, causing damage to its motors. The loss of one entity, as depicted in Figure \ref{fig:2Drop}, can potentially decrease the quality of execution, in case that the task of the fallen drone remain unattended. With this in mind, a new plan is required and must consider the new tasks that originally belong to the fallen drone. One possibility to avoid the problem is to change the current C2 approach, i.e., change the network configuration and leaders. Furthermore, if the team changes to an unsuitable C2 approach or is not even capable of changing its C2 approach, both cases represent a lack of C2 Maneuver Agility. The absence of a strategy to increase C2 Maneuver Agility can cause the system to compromise the number of completed tasks at the end of execution. 

\begin{figure}
\centering
\fbox{
\begin{minipage}{.45\textwidth}
  \centering
  \includegraphics[width=0.95\linewidth]{figures/C2Drones1-V4.png}
  \captionof{figure}{Team of entities with the coordinator marked in blue}
  \label{fig:2TeamExecution}
\end{minipage}}%
\fbox{
\begin{minipage}{.45\textwidth}
  \centering
  \includegraphics[width=0.95\linewidth]{figures/C2Drones2-V4.png}
  \captionof{figure}{Dynamic change with an UAV dropped marked in red}
  \label{fig:2Drop}
\end{minipage}}
\end{figure}

% Generalize the problem / Why is this not solved yet?
In general, some context changes, e.g., UAV failure, task addition, or sensor damage, must be considered during the mission planning. Indeed, no adaptation to the new context can impact quality results due to an incompatibility between entities and mission, or insufficient resources to complete the mission, or even the inability to meet minimum quality acceptance level. In other words, if there is limited C2 Maneuver Agility, the mission might be compromised.

Finally, Figure \ref{fig:3TeamExecutionAfterManuever} gives an example of a possible maneuvering strategy to maintain quality of execution. In this proposed example, immediately after the context change, the only member available for executing tasks type 4 is now offline. Moreover, the only member of the team capable of executing this new available task is the team's leader. Thus, after maneuvering to the \textit{Edge}~\citep{france2014} approach, the past leader can now participate in the execution of tasks, instead of only give orders to the team like in the previous strategy.

\begin{figure}[ht]
    \centering
    \fbox{
    \begin{minipage}{.45\textwidth}
  \centering
  \includegraphics[width=0.95\linewidth]{figures/C2Drones3-V4.png}
  \captionof{figure}{Team executing tasks after changed C2 Approach}
  \label{fig:3TeamExecutionAfterManuever}
\end{minipage}}
\end{figure}
\section{Channel System}
\label{sec:channelSystem}
\section{Design Implementation}
\label{sec:design}

Our proposed design, using program graphs and channel systems, models a group of entities coordinating to complete a common goal or objective while managing dynamic events. The team's members can be diverse, due to their different set of attributes and behaviours, according to their set of roles. Moreover, due to the heterogeneity of the group, each entity can react to various situations differently, in line with its information about the environment. Also, the communication topology among the team limits their interactions and can have a large impact on the overall system. With that in mind, our proposed design fits as an application of an agent-based model (ABM)~\cite{evaluating}. 

To implement our application of an ABM, we used a framework on the the Java platform, \textit{Repast Symphony}~\cite{repastDoc}. The aforementioned framework is commonly used for ABM simulations~\cite{repast} and it's capacity of simulating parallelism is mandatory in our domain, despite its implementation uses concurrent programming~\cite{repastDoc}. Finally, the object-oriented paradigm is appropriate to encapsulate each role, i.e, program graph separately, making it trivial to add or remove roles throughout the execution.



% ROLE-BASED MODELING SECTION

% Also, because we separate a group of reusable functionalities encapsulated in different roles, our approach also applies principles from role-based modeling~\cite{roleOrientedModeling}.

% Finally, separating a group of reusable functionalities in different roles, i.e, in different program graphs, is an implementation design of role-based modeling~\cite{roleOrientedModeling, modelingAgentOrganizationsUsingRoles}.

% Roles are used to form different interfaces for agents in order to restrict the visibility of features~\cite{roleOrientedModeling}, such as internal attributes. Concerning their internal behaviour roles may capture goals and handle responsibilities~\cite{roleOrientedModeling} to execute tasks autonomously. Additionally, as demonstrated in Section \ref{subsec:PG}, roles can be dynamically attached to and retracted from an agent. This feature is especially important if a role shall migrate from one agent to another~\cite{roleOrientedModeling} in run time. For instance, when a maneuver occurs some agents may undergo a complete transformation in its roles, depending on the role of the C2 approach selector to decide which set of roles is more appropriate for each agent.

% A role class can be described in terms of its breath and dept~\cite{modelingAgentOrganizationsUsingRoles}. Breadth, or horizontal specialization, addresses the number and complexity of actions supported by a given role~\cite{modelingAgentOrganizationsUsingRoles}. Depth, or vertical specialization, relates to the degree of control an agent can have over its actions and the actions of other agents~\cite{modelingAgentOrganizationsUsingRoles}. Recalling Section \ref{sec:introduction}, horizontal specialization is more related to C2 Approach Agility, due to its nature of providing agility within the same C2 approach, adding resilience and flexibility in the current structure of execution. Similarly, vertical specialization relates to C2 Maneuver Agility. Dept relates to the task allocator and C2 approach selector roles, due to their level of influence over the remaining members while performing a maneuver or even which tasks they will execute.

\newpage
\section{Evaluation}
\label{sec:evaluation}

To estimate the obtained C2 Maneuver Agility in the system, we collected evidence from the implementation based on the proposed model. Such simulation plays different scenarios that explore dynamic contexts, i.e.., the circumstances can change during the mission execution. These scenarios allow us to test the system effectiveness under such conditions applying the C2 concept. Thus, our goal with the simulations is to answer the following research question:

\begin{center}
\fbox{\begin{minipage}{25em}
\textit{How the C2 Maneuver Agility contributes with the C2 Agility?}
\end{minipage}}    
\end{center}

Next, we define the metrics applied in the evaluation to answer the proposed question and their corresponding descriptions:
\begin{itemize}
    \item Maneuvering (M1): Number of C2 Maneuvers performed by the members to accomplish the mission within a given timeout;
    \item Timeliness (M2): System time, in ticks, to accomplish the mission within a given timeout;
    \item Effectiveness (M3): Percentage of successful tasks completed by the executors.
\end{itemize}

We applied two different methods of response in case of context changes, so-called A1 and A2. Both methods starts the execution with an initial context predefined and it performs a task allocation. To the A1 method, the context changes, e.g., member dropped, or sensors' issue, does not initiate any system's reaction to deal with. In such case, the system just keeps running and can become ineffective with the new circumstance. In turn, the A2 method applies the C2 Agility concept to deal with these new conditions. This strategy includes the C2 Maneuver Agility represented by the computational model proposed in Section~\ref{sec:channelSystem} and whose implementation is described in Section~\ref{sec:design}.

The simulation based on an agent tool incorporates C2 Agility characteristics in face of the events created to simulate a dynamic scenario. According to this architecture, C2 Maneuver Agility looks for a suitable adoption of the communication structure to provide an information exchanging, which increases member's awareness. Such coordination structure, represented by the network topology played by the members, is relevant to the members get a suitable task allocation and reconfiguration, i.e., enabling and disabling its sensors in runtime. Thus, the A2 method applies both behaviors, i.e., C2 Agility, in response to new circumstances in the scenarios.

%%%%%%%%%%%%%%%%%%%%%%%%%%%%%%%%%%
% SUBSECTION 
%%%%%%%%%%%%%%%%%%%%%%%%%%%%%%%%%%
\subsection{Simulation Scenarios}
\label{sub:scenarios}

A simulation scenario is composed of an initial context and a sequence of events that provide dynamism. The initial context comprises the set $E$ of members operating an initial C2 Approach $\omega$, the mission $M$ composed of a set of tasks, and the environment. Each possible event, e.g., \textit{memberFailure}, \textit{sensorFailure}, and \textit{envChange}, represents an action that causes a member or environment changes in runtime. These actions are called during simulation to create a dynamic scenario, resembling realistic settings.

The environment represents all the conditions of the place where the members act, e.g., weather conditions, hazard, and communication restrictions. Onboard sensors' capability alteration represents such changes. Based on this, a foggy or cloudy day reduces the uselessness of VGA cameras depending on the task to be performed. Thus, all members with this type of sensors are impacted and it modifies the task allocation performed.

Table \ref{tab:scenarios} shows the sequences of actions that characterizes the context changes in dynamic scenarios. These sequences combined with the same initial context, i.e., members, mission, environment, and the initial C2 Approach,  define the scenarios manipulated by the simulation. These scenarios have a time limit of execution, and the events of environment changes follow the sequences showed. However, their turn is randomly within the simulation timeout.  

\begin{table}[h]
\centenring
\fontsize{9}{9}
\selectfont
\caption{List of events(\textit{EC-envChange; SF-sensorFailure; MF-memberFailure}) that characterizes the context changes within the scenarios tested. The initial C2 Approach, the set of members $E$ and the mission $M$ remain unchanged.}
\label{tab:scenarios}
\begin{tabular}{|m{0.1\textwidth}|m{0.84\textwidth}|}
\hline
\rowcolor{lightgray}
 \textbf {Scenario} & \hfil  \textbf {Context Changes} \\
\hline
 \hfil 1 & EC $\rightarrow$ EC $\rightarrow$ EC $\rightarrow$ EC $\rightarrow$ EC\\
\hline 
 \hfil 2 & EC $\rightarrow$ EC $\rightarrow$ EC $\rightarrow$ EC $\rightarrow$ EC $\rightarrow$ EC $\rightarrow$ EC $\rightarrow$ EC $\rightarrow$ EC $\rightarrow$ EC $\rightarrow$ EC $\rightarrow$ EC $\rightarrow$ EC $\rightarrow$ EC $\rightarrow$ EC \\
\hline 
 \hfil 3 & 
EC $\rightarrow$ EC $\rightarrow$ EC $\rightarrow$ EC $\rightarrow$ EC $\rightarrow$ MF $\rightarrow$ EC $\rightarrow$ EC $\rightarrow$ EC $\rightarrow$ EC $\rightarrow$ EC $\rightarrow$ MF $\rightarrow$ EC $\rightarrow$ EC $\rightarrow$ EC $\rightarrow$ EC $\rightarrow$ EC $\rightarrow$ MF $\rightarrow$ EC $\rightarrow$ EC $\rightarrow$ EC $\rightarrow$ EC $\rightarrow$ EC  \\
\hline 
 \hfil 4 & EC $\rightarrow$ EC $\rightarrow$ EC $\rightarrow$ EC $\rightarrow$ EC $\rightarrow$ SF $\rightarrow$ EC $\rightarrow$ EC $\rightarrow$ EC $\rightarrow$ EC $\rightarrow$ EC $\rightarrow$ SF $\rightarrow$ EC $\rightarrow$ EC $\rightarrow$ EC $\rightarrow$ EC $\rightarrow$ EC $\rightarrow$ SF $\rightarrow$ EC $\rightarrow$ EC $\rightarrow$ EC $\rightarrow$ EC $\rightarrow$ EC  \\
\hline 
 \hfil 5 & EC $\rightarrow$ EC $\rightarrow$ EC $\rightarrow$ SF $\rightarrow$ EC $\rightarrow$ EC $\rightarrow$ EC $\rightarrow$ MF $\rightarrow$ EC $\rightarrow$ EC $\rightarrow$ EC $\rightarrow$ SF $\rightarrow$ EC $\rightarrow$ EC $\rightarrow$ EC $\rightarrow$ MF $\rightarrow$ EC $\rightarrow$ EC $\rightarrow$ EC $\rightarrow$ SF $\rightarrow$ EC $\rightarrow$ EC $\rightarrow$ EC $\rightarrow$ MF  \\
\hline 
 \hfil 6 & EC $\rightarrow$ SF $\rightarrow$ MF $\rightarrow$ EC $\rightarrow$ SF $\rightarrow$ EC $\rightarrow$ SF $\rightarrow$ EC $\rightarrow$ SF $\rightarrow$ MF $\rightarrow$ EC $\rightarrow$ SF $\rightarrow$ EC $\rightarrow$ SF $\rightarrow$ EC $\rightarrow$ SF \\
\hline 
 \hfil 7 & MF $\rightarrow$ SF $\rightarrow$ EC $\rightarrow$ EC $\rightarrow$ EC $\rightarrow$ EC $\rightarrow$ EC $\rightarrow$ EC $\rightarrow$ EC $\rightarrow$ EC $\rightarrow$ EC $\rightarrow$ EC $\rightarrow$ MF $\rightarrow$ SF $\rightarrow$ EC $\rightarrow$ EC $\rightarrow$ EC $\rightarrow$ EC $\rightarrow$ EC $\rightarrow$ MF $\rightarrow$ SF $\rightarrow$ EC $\rightarrow$ EC $\rightarrow$ EC $\rightarrow$ MF $\rightarrow$ SF $\rightarrow$ EC \\
 \hline
 \hfil 8 & MF $\rightarrow$ SF $\rightarrow$ EC $\rightarrow$ MF $\rightarrow$ SF $\rightarrow$ EC $\rightarrow$ MF $\rightarrow$ SF $\rightarrow$ EC $\rightarrow$ MF $\rightarrow$ SF $\rightarrow$ EC \\
 \hline
 \hfil 9 & SF $\rightarrow$ SF $\rightarrow$ SF $\rightarrow$ MF $\rightarrow$ SF $\rightarrow$ SF $\rightarrow$ SF $\rightarrow$ MF $\rightarrow$ SF $\rightarrow$ SF $\rightarrow$ SF $\rightarrow$ MF $\rightarrow$ SF $\rightarrow$ SF $\rightarrow$ SF \\
 \hline
 \hfil 10 & SF $\rightarrow$ EC $\rightarrow$ MF $\rightarrow$ SF $\rightarrow$ EC $\rightarrow$ MF $\rightarrow$ SF $\rightarrow$ EC $\rightarrow$ MF $\rightarrow$ SF $\rightarrow$ EC $\rightarrow$ MF $\rightarrow$ EC $\rightarrow$ EC $\rightarrow$ EC $\rightarrow$ EC $\rightarrow$ EC $\rightarrow$ EC $\rightarrow$ EC $\rightarrow$ EC $\rightarrow$ EC $\rightarrow$ EC \\
 \hline
\end{tabular}

\end{table}


The simulation operates a scenario with 5 possible types of tasks (0 to 4) and 5 types of sensors (A, B, C, D, and E). The tasks and sensors onboard are randomly chosen before the running round. When the tasks allocation process starts, the algorithm applies a function that returns the quality $Q_{ij}$, obtained from a table, that correlates a sensor $i$ to the task $j$. When $Q_{ij}=0$ it means the sensor $i$ is not able to perform the task $j$.

A context change simulated by the system causes the quality reduction of a specific type of sensor, e.g., an environment change simulating a luminosity decreasing reduces in $50\%$ the quality of the sensor type 2 that represents a VGA camera. In case the sensor burns out, its quality comes to be zero and the task will be transmitted to another member to check its execution capacity. Also, the member with a burned sensor is capable of enabling and disabling sensors depending on the task it is currently executing. For instance, after the quality reduction of its sensor type 2, it disables the defective sensor while enabling the best sensor for its next target, present in its allocation list. Furthermore, we can lose a member and, consequently, all its sensors onboard.

Context changes can generate situations such as the tasks reallocation is not enough to keep mission execution. In that case, the C2 System performs a C2 Approach change, i.e.., maneuvering. The new C2 Approach operated can provide a higher awareness level, i.e.., more information shared by the members with a new communication structure, and it can help the members perform reallocation based on the new data exchanged. The initial C2 Approach for all scenarios simulated is De-Conflicted, i.e., a ring communication structure. From this C2 Approach, the maneuver follows incrementally over a specific order of C2 Approaches, based on the following research~\cite{france2014}, and going back to Conflicted after Edge. The Conflicted is the final C2 Approach adopted by the system before discarding the unfeasible tasks.

%%%%%%%%%%%%%%%%%%%%%%%%%%%%%%%%%%
% SUBSECTION 
%%%%%%%%%%%%%%%%%%%%%%%%%%%%%%%%%%
\subsection{Experimental Setup}
\label{sub:setup}

Figure \ref{fig:exp_setup} shows the experimental setup applied to assess C2 Agility, which is composed by C2 Maneuver Agility. A factorial experiment is performed with the treatments applied to the simulation scenario (Section~\ref{sub:scenarios}) and the methods A1 and A2. This factorial experiment was performed 500 times to obtain $95\%$ of confidence level~\cite{CochranW.G.1983}. Section~\ref{sec:evaluation} shows the metrics applied in the evaluation.

\begin{figure}[ht!]
    \centering
    \scalebox{.65}{\input{tikz/experimentalDesign}}
    \caption{Experimental setup with the treatments performed by the simulator}
    \label{fig:exp_setup}
\end{figure}

All scenarios created from the variables' values definition consider the same size of members' set $|E|=5$, the mission size $|M|=30$, and an initial C2 Approach $\omega_0 =$ De-Conflicted. Finally, each simulation scenario has a deadline of 1000 ticks of execution time.


%The simulation uses a set of random variables to define some elements during execution, i.e., UAVs, and tasks position, and sensor that will be affected by the context event. For a consistent comparison between the methods A1 and A2, the same set of random variables is used by both methods during the same execution number.

%%%%%%%%%%%%%%%%%%%%%%%%%%%%%%%%%%
% SUBSECTION 
%%%%%%%%%%%%%%%%%%%%%%%%%%%%%%%%%%
\subsection{Results and Analysis}

Table~\ref{table:results02} shows the results obtained by the simulation with the scenarios described in Section~\ref{sub:scenarios}. Figures~\ref{fig:m1},~\ref{fig:m2}, and~\ref{fig:m3} make a graphical comparison that shows a statistical difference among results obtained with the A1 and A2 treatments. Figure~\ref{fig:m1} confirms the maneuvering presence in all scenarios with the A2 treatment. The results are constant value due to the cost in terms of time to perform a C2 Approach change. As expected, the A1 method does not involve such behavior due to the nonexistence of C2 Maneuver Agility.

\begin{figure}[ht]
\centering
\begin{minipage}{.5\textwidth}
    \centering
    \small
    \fontsize{7}{7}\selectfont
	\captionof{table}{Metrics results}
    \label{table:results02}
    \begin{tabular}{rrrr} \hline
		& \bf{M1}
        & \bf{M2}
		& \bf{M3} \\  \hline 
		
		& Mean(St.Dev.)  & Mean(St.Dev.) & Mean(St.Dev.)\\ [1ex]
		
		\multicolumn{3}{l}{\textbf{$\longrightarrow$ Scenario 1 }} \\
	% Scenario  1 
\bf{A1}  & 0.0 ($\pm$0.0)  & 890.0 ($\pm$66.8)  & 82.0 ($\pm$5.9)  \\
\bf{A2}  & 1.7 ($\pm$0.3)  & 945.5 ($\pm$33.1)  & 97.7 ($\pm$2.9)  \\ [1ex]
	
	\multicolumn{3}{l}{\textbf{$\longrightarrow$ Scenario 2 }} \\
% Scenario  2 
\bf{A1}  & 0.0 ($\pm$0.0)  & 866.5 ($\pm$72.3)  & 77.5 ($\pm$6.3) \\
\bf{A2}  & 2.6 ($\pm$0.6)  & 934.0 ($\pm$33.0)  & 95.2 ($\pm$4.2) \\ [1ex]
	
	\multicolumn{3}{l}{\textbf{$\longrightarrow$ Scenario 3 }} \\
% Scenario  3 
\bf{A1}  & 0.0 ($\pm$0.0)  & 782.1 ($\pm$78.3)  & 63.7 ($\pm$5.5) \\
\bf{A2}  & 2.8 ($\pm$0.3)  & 883.0 ($\pm$49.2)  & 69.1 ($\pm$4.5) \\ [1ex]
	
	\multicolumn{3}{l}{\textbf{$\longrightarrow$ Scenario 4 }} \\
% Scenario  4 
\bf{A1}  & 0.0 ($\pm$0.0)  & 799.1 ($\pm$77.0)  & 63.6 ($\pm$5.3) \\
\bf{A2}  & 3.0 ($\pm$0.1)  & 913.4 ($\pm$35.9)  & 84.5 ($\pm$5.1) \\ [1ex]
	
	\multicolumn{3}{l}{\textbf{$\longrightarrow$ Scenario 5 }} \\
% Scenario  5 
\bf{A1}  & 0.0 ($\pm$0.0)  & 733.7 ($\pm$75.5)  & 54.6 ($\pm$4.6)  \\
\bf{A2}  & 2.9 ($\pm$0.2)  & 887.7 ($\pm$48.4)  & 70.7 ($\pm$5.1) \\ [1ex]
	
	\multicolumn{3}{l}{\textbf{$\longrightarrow$ Scenario 6 }} \\
% Scenario  6 
\bf{A1}  & 0.0 ($\pm$0.0)  & 700.6 ($\pm$79.4)  & 42.0 ($\pm$4.5) \\
\bf{A2}  & 2.9 ($\pm$0.2)  & 910.5 ($\pm$59.3)  & 65.7 ($\pm$5.9) \\ [1ex]
	
	\multicolumn{3}{l}{\textbf{$\longrightarrow$ Scenario 7 }} \\
% Scenario  7 
\bf{A1}  & 0.0 ($\pm$0.0)  & 686.2 ($\pm$58.4)  & 41.1 ($\pm$3.5)  \\
\bf{A2}  & 2.8 ($\pm$0.3)  & 820.5 ($\pm$61.4)  & 60.1 ($\pm$5.6)  \\ [1ex]
	
	\multicolumn{3}{l}{\textbf{$\longrightarrow$ Scenario 8 }} \\
% Scenario  8 
\bf{A1}  & 0.0 ($\pm$0.0)  & 629.7 ($\pm$68.6)  & 36.3 ($\pm$3.8) \\
\bf{A2}  & 2.8 ($\pm$0.3)  & 785.1 ($\pm$58.3)  & 52.1 ($\pm$4.7) \\ [1ex]
	
	\multicolumn{3}{l}{\textbf{$\longrightarrow$ Scenario 9 }} \\
% Scenario  9 
\bf{A1}  & 0.0 ($\pm$0.0)  & 491.0 ($\pm$74.2)  & 26.4 ($\pm$3.7) \\
\bf{A2}  & 3.0 ($\pm$0.1)  & 743.2 ($\pm$66.1)  & 53.2 ($\pm$5.1) \\ [1ex]
	
	\multicolumn{3}{l}{\textbf{$\longrightarrow$ Scenario 10 }} \\
		
% Scenario  10 
\bf{A1}  & 0.0 ($\pm$0.0)  & 393.4 ($\pm$81.2)  & 24.3 ($\pm$2.6) \\
\bf{A2}  & 2.9 ($\pm$0.2)  & 650.7 ($\pm$114.8)  & 38.8 ($\pm$4.7) \\ [1ex]
		\hline
	\end{tabular}
\end{minipage}%
\begin{minipage}{.5\textwidth}
  \centering
  \includegraphics[width=0.95\linewidth]{figures/graphs/Boxplot_M2.png}
  \captionof{figure}{Total of maneuverings (M1)}
  \label{fig:m1}
\end{minipage}
\end{figure}

\begin{figure}
\centering
\begin{minipage}{.5\textwidth}
  \centering
  \includegraphics[width=0.95\linewidth]{figures/graphs/Boxplot_M3.png}
  \captionof{figure}{Total operation time (M2)}
  \label{fig:m2}
\end{minipage}%
\begin{minipage}{.5\textwidth}
  \centering
  \includegraphics[width=0.95\linewidth]{figures/graphs/Boxplot_M5.png}
  \captionof{figure}{Completed tasks (M3)}
  \label{fig:m3}
\end{minipage}
\end{figure}

Similar analysis can be applied to M1 and M2 metrics. Figure~\ref{fig:m2} shows significant difference among the time in operation, i.e., timeliness, of the methods A1 and A2. Complex scenarios with more events causing context changes, the system with C2 Agility keeps running for longer. In a such condition, the awareness obtained by the system through the C2 Approach selected becomes the system more suitable to deal with a new circumstance and capable to keep engaged in the mission. In addition, the metric M3 (Figure~\ref{fig:m3}) shows a better mission completeness when the system makes coordination and adaptation adjustments in runtime in face of context changes. 

More time consumed by the system could be related to the cost in terms of time to perform such maneuvers. However, A combined analysis of the results showed in Figures~\ref{fig:m2} and ~\ref{fig:m3}, indicates a capacity to keep running under context perturbations and performing the mission tasks. It gives more resilience to the system achieving C2 Agility and, consequently, C2 Maneuver Agility. Although sensor reconfigurations are happening in all scenarios due to C2 Agility implemented, those with higher count of maneuvers values (M1) pointed out better results (M3) when compared with the A1 method. This aspect shows evidence of gains from the incorporation of C2 Maneuver agility in the implementation.

As the system gets different information sharing level through C2 Approach changes, i.e., maneuverings, it can adapt itself according to the different conditions that occur in a dynamic scenario. The Figure~\ref{fig:m3} shows that the capacity of the C2 system to solve tasks, under the effect of changes in the context that alter its functioning, is more significant in the A2 method where C2 Maneuver Agility provides the system awareness modification. In turn, the static nature of the A1 method limits its performance since the initial configuration is the only and unique option to execute the mission.

All results obtained in our experiment were submitted to a Shapiro-Wilk test~\citep{stat001}. Such a test gave us a \textit{p-value} less than 0.05 and it indicates a non-normal distribution. Based on this, we applied a Mann-Whitney U Test~\citep{stat002}, available in R tool as Wilcoxon-test, to check statistical differences among the samples of each action method applied. The statistical analysis confirmed the difference between all results from A1 and A2 methods, collected in all scenarios simulated.


% In summary, the empirical findings indicate a better overall performance using the A2 method that implements C2 Agility. This capability is only completed with the C2 Maneuver Agility concept, where the system looks for a suitable coordination structure, i.e., C2 Approach, to deal with context changes. Results show a relation between the number of C2 Approach changes and the number of completed tasks, indicating its contribution to keeping the system engaged with the mission. These maneuverings, i.e., C2 Approach changes, have a cost in terms of time, and even with this, the time in operation, i.e., M2 metric, increases along with the mission completeness(M3). Besides, the model(CS) used to represent the entities' behavior that makes up the system, through the use of roles, proved to be effective in guiding the implementation of the simulator. The capabilities embedded in the system result in a resilience increasing and enable the behavioral model the ability to cope with context changes and perturbations in dynamic scenarios.

In summary, the empirical findings indicate a better overall performance using the A2 method that implements C2 Agility. This capability was achievable with the presence of the C2 Maneuver Agility concept, where the system looks for a suitable coordination structure, i.e., C2 Approach, to deal with context changes. Results show a relation between the number of C2 Approach changes and the number of completed tasks, that may indicate a contribution of maneuvering to keeping the system engaged with the mission. These maneuverings, i.e., C2 Approach changes, have a cost in terms of time, and even with this, the time in operation, i.e., M2 metric, increases along with the mission completeness (M3). Besides, the proposed communication model (CS) along side with the adoption of roles (PG) used to represent the entities' behavior proved to be effective in guiding the implementation of the simulator. The capabilities embedded in the system result in a resilience increasing and enable the behavioral model the ability to cope with context changes and perturbations in dynamic scenarios.


%<<< 
%PRECISAMOS ADICIONAR: 
% -> o modelo comportamental (CS) pôde representar a coordenação entre os papeis que modelam o cenário de C2 -> missing
% -> O C2 Maneuver Agility compõe a definição de C2 Agility e sua implementação possibilita a adequada troca de informações em busca da awareness do sistema, de modo que ele se adeque aos diversos contextos (Não precisamos falar de configuração dos elementos) -> +/-
% -> evidencias -> accomplished
%>>>
\section{Conclusion}
\label{sec:conclusion}

Using the evaluation data (Section \ref{sec:evaluation}) as our base, we can conclude that providing more options to the agents, with different C2 approaches through maneuvers, increased performance in execution. This capability brings resilience and robustness to the overall system, due to the treatment of unexpected events by changing patterns of interactions, decision rights, and distribution of information, i.e, changing the team's organization.

The proposed design (Section \ref{sec:design}) has covered a lot of nuances in the simulation, e.g, sensor failure, drop of members. Sudden changes along execution were perceptible to the program graphs and were properly communicated to the right members using the channel systems, avoiding a significant drop in quality of the mission. Further, the ABM's roles concept allowed the system to change its configuration, necessary to perform maneuvers.

Furthermore, as future work, an evolution of the simulation would be to add more types of dynamic events, such as changing the mission, i.e, adding or removing tasks during execution, to measure the performance of the proposed design with a wider range of context changes. Moreover, because of the implementation's modularity nature, other functions can be added to allocate tasks or detect the need for maneuvering depending on some criteria, e.g, if energy and duration are not a problem, artificial intelligence and big data can be used to allocate tasks. Also, different metrics can be used to evaluate the model, such as resilience (\textit{How can we maintain our current level of performance?}) and quality of tasks execution (\textit{How to execute tasks balancing best sensor to the task and responsiveness?}).

In conclusion, evidence provided by the simulation indicates that the presented design seems adequate to specify problems in the Command and Control domain. The model seems to provide enough awareness to detect context changes and it equips the system with options of how to deal with them by increasing the C2 maneuver agility level.

%\section*{\refname}
%\bibliography{references}
\bibliography{references}
%qualifying

%\newpage
%\begin{appendices}
%
%\renewcommand\thefigure{\thesection.\arabic{figure}}
%\setcounter{figure}{0}
%
%\section{Probabilistic Models}
%\label{app:probabilistic-models}
%\input{content/appendix/models}
%
%\end{appendices}

\end{document}
