\documentclass[acmsmall,colorlinks, dvipsnames]{acmart}
%% colorlinks: option for package hyperref
%% dvipsnames: option for package xcolor

\usepackage{mystyle}
\usepackage{mymacros}
\usepackage{booktabs}
\usepackage{xtab}

% Stretch whitespace a little bit to avoid overfull.
\emergencystretch=1em

\setcitestyle{numbers, comma}

%%Este comando eh usado para referenciar código PVS
% \newcommand{\pvscode}[1]{\texttt{#1}} 
\newcommand{\pvscode}[1]{\ensuremath{\mathit{#1}}} 
\newtheorem{axiom}{Axiom}

%\journal{Science of Computer Programming}

%%
%% These commands are for a JOURNAL article.
\acmJournal{TOSEM}

%% The following commands result in compilation errors if given empty arguments.
% \acmVolume{}
% \acmNumber{}
% \acmArticle{}
% \acmMonth{}



%\bibliographystyle{elsarticle-harv}
\bibliographystyle{model2-names}


\begin{document}
 
\selectlanguage{english}

%\begin{frontmatter}



\title{Analysis of Command and Control Strategy Through the Interaction of Software Components}

    % Authors
    
%Thiago
%Leopoldo
%Vander
%Sven
%Maxime
%Rohit

\author{Eduardo Lemos Rocha}
\affiliation{%
  \institution{University of Brasília}
  \streetaddress{Campus Universitário Darcy Ribeiro - Edifício CIC/EST, 70910-900}
  \city{Brasilia}
  \country{Brazil}}
\email{dudulr10@gmail.com}

\author{Vander Alves}
\affiliation{%
  \institution{University of Brasília}
  \streetaddress{Campus Universitário Darcy Ribeiro - Edifício CIC/EST, 70910-900}
  \city{Brasilia}
  \country{Brazil}}
\email{valves@unb.br}


%%
%% The abstract is a short summary of the work to be presented in the
%% article.
\begin{abstract}

--

\end{abstract}

%%
%% The code below is generated by the tool at http://dl.acm.org/ccs.cfm.
%%
\begin{CCSXML}
<ccs2012>
   <concept>
       <concept_id>10011007.10011074.10011092.10011096.10011097</concept_id>
       <concept_desc>Software and its engineering~Software product lines</concept_desc>
       <concept_significance>500</concept_significance>
       </concept>
   <concept>
       <concept_id>10011007.10011074.10011099.10011692</concept_id>
       <concept_desc>Software and its engineering~Formal software verification</concept_desc>
       <concept_significance>500</concept_significance>
       </concept>
 </ccs2012>
\end{CCSXML}

\ccsdesc[500]{Software and its engineering~Software product lines}
\ccsdesc[500]{Software and its engineering~Formal software verification}

\keywords{Software Product Lines, Product-line Analysis, Framework, Formalization}

%%
%% This command processes the author and affiliation and title
%% information and builds the first part of the formatted document.
\maketitle
    

%    \input{content/abstract}

%    \begin{keyword}
%        Software Product Lines\sep Product-line Analysis \sep Verification
%    \end{keyword}

%\end{frontmatter}

\section{Introduction}
\label{sec:introduction}

\textit{Command and Control (C2)} is about focusing the efforts of a set of entities and resources towards the achievement of some task, objective, or goal~\citep{CC02}. The entities may represent individuals or organizations, and resources involve everything manipulated by the entities, including information exchange. Originally developed in military domain, C2 was based on the idea of a central command concentrating information and power over required elements to accomplish the mission~\citep{CC01}.

In the information age, C2 has developed to absorb the new resources brought by modern technologies in the information exchange process. The result of these developments has been the application of C2 in several domains, such as by Civil Defense during disaster relief operations, and financial operations managing resources to maximize results~\citep{CC03,CC04}. Further C2 applications include sensitive issues as nuclear weapon and research~\citep{C2-EX2}, national mass-vaccination campaigns~\citep{C2-EX1}, and the COVID-19 pandemic scenario, orchestrating different government and research organizations to find a mitigation or solution to the problem and providing responses to the society~\citep{C2-EX3, C2-EX4, C2-EX5}.

The complexity of existing scenarios in different C2 domains stems mostly from the inherent dynamism present in all of them due to changes of circumstance or context. For instance, the sudden lack of entities or the increased risk of the scenario are some of the problems that characterize circumstance changes of C2 applied in the Civil Defense domain. Such scenario changes characterize context dynamism, which can occur in the mission, environment, or entity.

\textit{C2 agility} is the entity's capability of dealing with context changes~\citep{france2014}. To cope with some of these, providing the entities with new resources has been reported to suffice~\citep{france2014}. On the other hand, more complex scenarios caused by new circumstances require changing the collaboration approach among entities to deal with this dynamism in a suitable way. Essentially, C2 agility is a timely and effective response to context changes.

Nevertheless, there is a lack of evidence on how to provide C2 agility. Indeed, state-of-the-art and state-of-the-practice approaches rarely explore context changes. The few that do focus on randomized network-level reconfiguration rather than on entity-level reconfiguration and thus do not deal adequately with the adaptation of the mission accomplishment, which does not characterize the C2 agility.~\citep{france2014,Alberts2017,evaluating,Alberts2011, nato01}. On the other hand, dynamism abounds in real scenarios  and thus are under circumstance changes all the time~\cite{evaluating}. This aspect highlights C2 agility's relevance to deal with context changes providing capability of adaptation. In this paper, we address this problem within the scope of simulated environments, which is often addressed in C2~\cite{CC02} in general and in its application areas such as the military domain~\cite{france2014}.

To provide C2 agility, we present a computational model that coordinates and reconfigures entities to handle context changes. Such model is a typed-parameterized extension of a channel system~\citep{modelcheckingBaier}. This extension, hereafter referred to as CS, defines the roles and responsibilities that are executed by the modelled entities. Each member executing roles is modelled as a Dynamic Software Product Line (DSPL)~\citep{Bencomo2008}. To cope with context changes, members can reconfigure themselves or change their coordination structure thereby achieving C2 agility.

Although modern techniques of Model Checking support a large number of states~\citep{modelcheckingBaier}, concurrent systems like ours may suffer from the possibility of states number explosion. \cite{clarkson2014} showed a prototype of Model Checker to QPTL~\citep{QPTL001} resulted from a transformation of HyperCTL formula. However the checker is impractical for real-world programs and does not provide a timely response in runtime when the system is under a dynamic context. Based on this, we evaluate the proposed C2 computational model performing an agility assessment in a software-based simulated environment.

Even the simulation environment being simpler than real scenarios, it is relevant to a myriad of different domains such as the military~\citep{CC03} and environmental monitoring~\cite{simulation001}. Simulation creates different circumstances to evaluate a solution or product, or to train professionals, reducing the need for resources to create real circumstances. Furthermore, many scenarios occur naturally in virtual environment, such as Network Centric Warfare and telemedicine, making simulations relevant to represent events that occur in virtual world
~\citep{telemedicine01, france2014, CC01}.

In our study, we simulate a set of UAVs applied in a reconnaissance mission. The simulation explores different scenarios with context changes. Such changes occur in the mission due to eventually tasks addition, in the entities due to aleatory damages in onboard  entities’ components and parts, or in the environment due to hazard increasing or weather conditions changes, thus causing impacts on the execution. The results of quantitative and qualitative agility metrics indicate that the entities have a certain level of agility. We also identify challenging situations in achieving agility and discuss related tradeoffs. In summary, this work makes the following contributions:

\begin{itemize}
    \item We present a typed-paramaterized channel system modelling C2 system roles, their interactions, and dealing with context changes (Section~\ref{sec:channelSystem});
    \item We design and implement the proposed  channel system and make it  publicly available\footnote{http://github.com/c2} (Section~\ref{sec:designImpl});
    \item We perform a simulation-based study to empirically evaluate the proposed computational model in providing C2 agility, according to quality and quantity metrics (Section~\ref{sec:evaluation}). 
\end{itemize}

\section{Motivating Example}
\label{sec:motivation}

% Context and Explanation of the example
To provide intuition about the problem, we present an exemplification of a mission execution by entities in a military context. Figure \ref{fig:2TeamExecution} illustrates a reconnaissance mission with some tasks randomly distributed requiring different types of sensors. In this mission, a team comprising four Unmanned Aerial Vehicles (UAVs) is interacting in a network configuration with a central leader, which is responsible only for providing instructions to its subordinates, e.g, distributing tasks among members. These interactions and network topology configure a specific C2 approach, named as \textit{Coordinated}~\citep{france2014}, based on the centralized distribution of information, patterns of interaction, and decision rights. The leader, marked in blue, guides the other team members to complete the tasks, i.e., obtain aerial images of particular points in the field, represented by red crosses.

% Highlighting the problem
The mission involves some natural risks that may lead to change in the conditions of execution. In our example, one member of the team, marked in red, fell due to some environmental change, such as an intense storm, causing damage to its motors. The loss of one entity, as depicted in Figure \ref{fig:2Drop}, can potentially decrease the quality of execution, in case that the task of the fallen drone remain unattended. With this in mind, a new plan is required and must consider the new tasks that originally belong to the fallen drone. One possibility to avoid the problem is to change the current C2 approach, i.e., change the network configuration and leaders. Furthermore, if the team changes to an unsuitable C2 approach or is not even capable of changing its C2 approach, both cases represent a lack of C2 Maneuver Agility. The absence of a strategy to increase C2 Maneuver Agility can cause the system to compromise the number of completed tasks at the end of execution. 

\begin{figure}
\centering
\fbox{
\begin{minipage}{.45\textwidth}
  \centering
  \includegraphics[width=0.95\linewidth]{figures/C2Drones1-V4.png}
  \captionof{figure}{Team of entities with the coordinator marked in blue}
  \label{fig:2TeamExecution}
\end{minipage}}%
\fbox{
\begin{minipage}{.45\textwidth}
  \centering
  \includegraphics[width=0.95\linewidth]{figures/C2Drones2-V4.png}
  \captionof{figure}{Dynamic change with an UAV dropped marked in red}
  \label{fig:2Drop}
\end{minipage}}
\end{figure}

% Generalize the problem / Why is this not solved yet?
In general, some context changes, e.g., UAV failure, task addition, or sensor damage, must be considered during the mission planning. Indeed, no adaptation to the new context can impact quality results due to an incompatibility between entities and mission, or insufficient resources to complete the mission, or even the inability to meet minimum quality acceptance level. In other words, if there is limited C2 Maneuver Agility, the mission might be compromised.

Finally, Figure \ref{fig:3TeamExecutionAfterManuever} gives an example of a possible maneuvering strategy to maintain quality of execution. In this proposed example, immediately after the context change, the only member available for executing tasks type 4 is now offline. Moreover, the only member of the team capable of executing this new available task is the team's leader. Thus, after maneuvering to the \textit{Edge}~\citep{france2014} approach, the past leader can now participate in the execution of tasks, instead of only give orders to the team like in the previous strategy.

\begin{figure}[ht]
    \centering
    \fbox{
    \begin{minipage}{.45\textwidth}
  \centering
  \includegraphics[width=0.95\linewidth]{figures/C2Drones3-V4.png}
  \captionof{figure}{Team executing tasks after changed C2 Approach}
  \label{fig:3TeamExecutionAfterManuever}
\end{minipage}}
\end{figure}
\section{Channel System}
\label{sec:channelSystem}
\section{Design Implementation}
\label{sec:design}

Our proposed design, using program graphs and channel systems, models a group of entities coordinating to complete a common goal or objective while managing dynamic events. The team's members can be diverse, due to their different set of attributes and behaviours, according to their set of roles. Moreover, due to the heterogeneity of the group, each entity can react to various situations differently, in line with its information about the environment. Also, the communication topology among the team limits their interactions and can have a large impact on the overall system. With that in mind, our proposed design fits as an application of an agent-based model (ABM)~\cite{evaluating}. 

To implement our application of an ABM, we used a framework on the the Java platform, \textit{Repast Symphony}~\cite{repastDoc}. The aforementioned framework is commonly used for ABM simulations~\cite{repast} and it's capacity of simulating parallelism is mandatory in our domain, despite its implementation uses concurrent programming~\cite{repastDoc}. Finally, the object-oriented paradigm is appropriate to encapsulate each role, i.e, program graph separately, making it trivial to add or remove roles throughout the execution.



% ROLE-BASED MODELING SECTION

% Also, because we separate a group of reusable functionalities encapsulated in different roles, our approach also applies principles from role-based modeling~\cite{roleOrientedModeling}.

% Finally, separating a group of reusable functionalities in different roles, i.e, in different program graphs, is an implementation design of role-based modeling~\cite{roleOrientedModeling, modelingAgentOrganizationsUsingRoles}.

% Roles are used to form different interfaces for agents in order to restrict the visibility of features~\cite{roleOrientedModeling}, such as internal attributes. Concerning their internal behaviour roles may capture goals and handle responsibilities~\cite{roleOrientedModeling} to execute tasks autonomously. Additionally, as demonstrated in Section \ref{subsec:PG}, roles can be dynamically attached to and retracted from an agent. This feature is especially important if a role shall migrate from one agent to another~\cite{roleOrientedModeling} in run time. For instance, when a maneuver occurs some agents may undergo a complete transformation in its roles, depending on the role of the C2 approach selector to decide which set of roles is more appropriate for each agent.

% A role class can be described in terms of its breath and dept~\cite{modelingAgentOrganizationsUsingRoles}. Breadth, or horizontal specialization, addresses the number and complexity of actions supported by a given role~\cite{modelingAgentOrganizationsUsingRoles}. Depth, or vertical specialization, relates to the degree of control an agent can have over its actions and the actions of other agents~\cite{modelingAgentOrganizationsUsingRoles}. Recalling Section \ref{sec:introduction}, horizontal specialization is more related to C2 Approach Agility, due to its nature of providing agility within the same C2 approach, adding resilience and flexibility in the current structure of execution. Similarly, vertical specialization relates to C2 Maneuver Agility. Dept relates to the task allocator and C2 approach selector roles, due to their level of influence over the remaining members while performing a maneuver or even which tasks they will execute.

\input{content/evalutation}
\section{Conclusion}
\label{sec:conclusion}

Using the evaluation data (Section \ref{sec:evaluation}) as our base, we can conclude that providing more options to the agents, with different C2 approaches through maneuvers, increased performance in execution. This capability brings resilience and robustness to the overall system, due to the treatment of unexpected events by changing patterns of interactions, decision rights, and distribution of information, i.e, changing the team's organization.

The proposed design (Section \ref{sec:design}) has covered a lot of nuances in the simulation, e.g, sensor failure, drop of members. Sudden changes along execution were perceptible to the program graphs and were properly communicated to the right members using the channel systems, avoiding a significant drop in quality of the mission. Further, the ABM's roles concept allowed the system to change its configuration, necessary to perform maneuvers.

Furthermore, as future work, an evolution of the simulation would be to add more types of dynamic events, such as changing the mission, i.e, adding or removing tasks during execution, to measure the performance of the proposed design with a wider range of context changes. Moreover, because of the implementation's modularity nature, other functions can be added to allocate tasks or detect the need for maneuvering depending on some criteria, e.g, if energy and duration are not a problem, artificial intelligence and big data can be used to allocate tasks. Also, different metrics can be used to evaluate the model, such as resilience (\textit{How can we maintain our current level of performance?}) and quality of tasks execution (\textit{How to execute tasks balancing best sensor to the task and responsiveness?}).

In conclusion, evidence provided by the simulation indicates that the presented design seems adequate to specify problems in the Command and Control domain. The model seems to provide enough awareness to detect context changes and it equips the system with options of how to deal with them by increasing the C2 maneuver agility level.

\section*{Acknowledgements}

We would like to thank the following people for fruitful discussions and suggestions on how to improve this work: Tobias Sena, Andreas Stahlbauer, Christoph Seidl, Malte Lochau, Matthias Kowal, Ina Schaefer, Thomas Th{\"u}m, and Pierre-Yves Schobbens. 
Vander Alves was partially supported by CNPq (grant 310757/2018-5), FAPDF (grant SEI 00193-00000926/2019-67), and the Alexander von Humboldt Foundation.
Leopoldo Teixeira was partially supported by CNPq (grant 409335/2016-9) and FACEPE (APQ-0570-1.03/14), as well as INES 2.0,\footnote{\url{http://www.ines.org.br}} FACEPE grants PRONEX APQ-0388-1.03/14 and APQ-0399-1.03/17, and CNPq grant 465614/2014-0.
Sven Apel was partially supported by the German Research Foundation (DFG) within the Heisenberg Programme (AP 206/6).

%\section*{\refname}
%\bibliography{references}
\bibliography{references}
%qualifying

%\newpage
%\begin{appendices}
%
%\renewcommand\thefigure{\thesection.\arabic{figure}}
%\setcounter{figure}{0}
%
%\section{Probabilistic Models}
%\label{app:probabilistic-models}
%\input{content/appendix/models}
%
%\end{appendices}

\end{document}
