\section{Introduction}
\label{sec:introduction}

% Context
Software product line engineering (SPLE) is a means to systematically
manage variability and commonality in software systems, enabling automated
synthesis of product variants from a set of reusable assets
\citep{ClementsSPL2001, PohlSPLE, FOSPL}.
SPLE aims at improving productivity and time-to-market, as well as
achieving mass customization of software \citep{PohlSPLE}.
% Problem
However, for a product line with high variability, the number of possibly generated products may be intractably large. Because of
this phenomenon, it is often infeasible to quality-check each of the products individually. 
Nonetheless, software analysis and verification techniques for single products are widely used by the industry, and it is beneficial to exploit their maturity to increase reliability while reducing costs and risks.


%Relevance
There is a number of approaches that lift analyses such as type checking, model checking, and theorem proving 
to the level of entire product lines~\cite{Thum2014}. 
\textit{Product-line analyses} can be classified along three strategy dimensions:
product-based (the analysis is performed on generated products or models thereof),
family-based (only domain artifacts and valid combinations thereof are checked),
and feature-based (domain artifacts implementing a given feature are analyzed in isolation,
regardless of their combinations)~\cite{Thum2014}.

Although different in purpose,  approaches to product-line analysis share features that suggest there is a yet unexplored potential for reuse of key analysis steps and properties, implementation, and verification efforts. 
%More than one dimension can be exploited in a given technique, giving rise
%to feature-family-based analyses (features are partially analyzed in isolation and 
%then the intermediate results are combined in a family-based fashion) and 
%family-product-based analyses (domain artifacts are partially analyzed
%considering only valid configurations, yielding a result that is prone to
%enumerative analysis), for instance.
%TODO @lmt: mention that if we explore the reuse, we might be able to derive other analyses?
%We can note unexplored similarities between a number of analysis approaches proposed so far. 
For instance, \citet{kowal_scaling_2015} employ a family-based analysis strategy for building a stochastic model that encodes all variability of the product line, and thus can be analyzed in a single run.
\citet{ApelSimulator} adopt the same strategy for generating source code of a program that encodes all variability in the product line as runtime variability, so that off-the-shelf software model checkers
can be applied. Although the models and verified properties are different, there
 is a pattern of encoding all variability in a single product, which \citet{ApelSimulator} call a product or variant simulator. This pattern, in turn, must have an underlying principle governing its
applicability and validity~\cite{variationalDataStructure}.
%TODO @lmt: there is the 150% model terminology as well, that we can use.
Therefore, should one have the need of applying these and similar analyses implemented by different tools, a number of analysis steps---that are inherently time-consuming, error-prone, and require specialized knowledge to perform---would inevitably be repeated. 

%Moreover, the separate development of analysis methods for software product lines yields
%unrelated tools, even if they take similar models as input or perform
%computations over the same data structures. This can lead to redundant
%implementations and even to repetition in correctness proofs. For example, \citet{Ghezzi2013} use UML sequence diagrams annotated with reliability and energy consumption values to analyze the reliability and energy consumption of a product line as a whole.
%In contrast, \citet{kowal_scaling_2015} propose a method that takes UML activity diagrams annotated with performance as input to evaluate performance of the overall product line. As both take UML behavioral models as input and verify stochastic properties, there is an opportunity for investigating
%reuse, even though they employ different variability handling techniques and perform different analyses.

As research in this area progresses, different analysis strategies may be applied to the same property. As an example, 
 \citet{Ghezzi2013} first explored compositionality of probabilistic models to analyze product-line reliability, thus proposing a feature-product analysis strategy. \citet{LANNA2017} combined compositionality while avoiding enumerating all products in a feature-family strategy.
Nevertheless, for the sake of soundness, each strategy carries the burden of proving that it yields the same results. This is often overlooked, error-prone, and time-consuming to achieve, eventually partially achieved with incomplete methods (e.g., testing~\cite{SPLLift}), ultimately jeopardizing confidence in such analysis strategies.


%
%% Solution
%Thus, we propose to apply SPL techniques to building a framework for
%product line analyses.
%% Evaluation
%This product line of SPL analysis techniques will be evaluated with respect
%to reuse and implementation effort,
%% Contributions
%by which we expect to gather information regarding commonality and variability
%in this domain. We also expect to gain insight on the underlying principles
%supporting the correctness of each of the implemented methods.
%


%\section{Problem Statement}
%\label{sec:problem}

The aforementioned repetition and formalization gaps are a natural outcome of ongoing research activity, given that
there are independent groups exploring similar issues. Nonetheless,
practitioners and researchers would benefit from having an integrated
body of knowledge. So, it is useful to periodically synthesize
existing work in a given field.
Indeed, the survey by \citet{Thum2014}, in which the terminology of family-, feature-, and product-based analysis strategies were established, is an important step
in this direction, which could be seen as a coarse-grained domain analysis of the domain of product-line analysis.
However, there still remains the underlying problem of not being able to describe precisely and uniformly product-line analyses and their properties. Indeed, more refinement is necessary to formalize concepts, to explore similarities in analysis steps and key properties, and to 
suitably manage variability among these analysis approaches, thus providing
practitioners and researchers with more efficient techniques and a theoretical framework for further investigation.



%Furthermore, there is no guidance on whether analysis strategies can be
%arbitrarily chosen for a given product line and analysis technique.
%\cite{Thum2014} state that the existence of a principle and possibly
%automated way to lift a given specification and analysis technique to product
%lines is an open research question. Accordingly, \cite{Midtgaard2015} proposed
%an approach to perform such lifting in the context of static analyses,
%reaching a mathematical formalism that justifies correctness of family-based
%analyses from correctness of product-based ones by construction. Nonetheless,
%it does not investigate feature-based strategies or other analysis types (e.g.,
%model checking).

%TODO @lmt: it might be worth mentioning Midtgaard's work in the introduction.

We propose a formal framework of key concepts and properties of product-line analyses. 
In particular, the framework defines abstract functions and types modeling essential abstractions in this problem domain, including central analysis steps, models, and intermediate analysis results.
Product-line analyses are  presented in a compositional manner, providing an overall understanding of the structure space of family-based, feature-based, and product-based analysis strategies, showing how the different types of product-line analyses compose and inter-relate.  
To ensure soundness, we formalize the framework using the PVS proof assistant~\citep{PVS:language}, providing a precise specification of key concepts and properties as well as proofs of key soundness results, e.g., commutativity of intermediate analysis steps. Therefore, the novelty of this work lies in how we model and map the different strategies and how we prove certain properties.
To qualitatively assess the generality of the framework, we discuss to what extent it is able to describe five representative product-line analyses on the following properties: safety, performance, data-flow facts, security, and functional program properties.
Analytical assessment by instantiation of the PVS theory for concrete analysis techniques is a promising avenue of 
future work. Our framework lays the foundation for this.

Overall, the formal framework encodes knowledge on the product-line analysis domain in a concise, precise, and sound manner, thus providing unambiguous definitions of domain terminology and assumptions as well as solid evidence of key domain properties based on formal proofs. This way, the framework provides principled understanding and further facilitates the communication of ideas and knowledge in this domain. Researchers and practitioners can safely leverage the framework to lift existing single-product analysis techniques to yet under-explored product-line analysis approaches and to explore further strategies in the latter.

In summary, the contributions of this article are the following:
\begin{itemize}
    \item We review existing product-line analyses, revealing their key concepts and properties  (Section~\ref{sec:strategies});
	\item We present a formal framework of product-line analyses that provides an overall understanding of the space of family-based, feature-based, and product-based analysis strategies as well as relations among them (Section~\ref{sec:generalStructure});
	\item We provide a mechanized formalization of the framework using the PVS proof assistant, comprising specification of essential concepts and proofs of key soundness results (Section~\ref{sec:abstraction-description});
%	\item We define a product line of product-line analysis strategies, built on top of the framework to systematically guide its customization (Section~\ref{sec:pl2ana});
	\item We explore the framework's generality, discussing to what extent it can describe five representative product-line analyses (Section~\ref{sec:evaluation}).
\end{itemize}

