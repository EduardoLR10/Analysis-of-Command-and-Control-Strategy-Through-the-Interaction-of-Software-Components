\section{Introduction}
\label{sec:introduction}

\textit{Command and Control (C2)} is about focusing the efforts of a set of entities and resources towards the achievement of some task, objective, or goal~\citep{CC02}. The entities may represent individuals or organizations, and resources involve everything manipulated by the entities, including information exchange. Originally developed in military domain, C2 was based on the idea of a central command concentrating information and power over required elements to accomplish the mission~\citep{CC01}.

In the information age, C2 has developed to absorb the new resources brought by modern technologies in the information exchange process. The result of these developments has been the application of C2 in several domains, such as by Civil Defense during disaster relief operations, and financial operations managing resources to maximize results~\citep{CC03,CC04}. Further C2 applications include sensitive issues as nuclear weapon and research~\citep{C2-EX2}, national mass-vaccination campaigns~\citep{C2-EX1}, and the COVID-19 pandemic scenario, orchestrating different government and research organizations to find a mitigation or solution to the problem and providing responses to the society~\citep{C2-EX3, C2-EX4, C2-EX5}.

The complexity of existing scenarios in different C2 domains stems mostly from the inherent dynamism present in all of them due to changes of circumstance or context. For instance, the sudden lack of entities or the increased risk of the scenario are some of the problems that characterize circumstance changes of C2 applied in the Civil Defense domain. Such scenario changes characterize context dynamism, which can occur in the mission, environment, or entity.

\textit{C2 agility} is the entity's capability of dealing with context changes~\citep{france2014}. To cope with some of these, providing the entities with new resources has been reported to suffice~\citep{france2014}. On the other hand, more complex scenarios caused by new circumstances require changing the collaboration approach among entities to deal with this dynamism in a suitable way. Essentially, C2 agility is a timely and effective response to context changes.

Nevertheless, there is a lack of evidence on how to provide C2 agility. Indeed, state-of-the-art and state-of-the-practice approaches rarely explore context changes. The few that do focus on randomized network-level reconfiguration rather than on entity-level reconfiguration and thus do not deal adequately with the adaptation of the mission accomplishment, which does not characterize the C2 agility.~\citep{france2014,Alberts2017,evaluating,Alberts2011, nato01}. On the other hand, dynamism abounds in real scenarios  and thus are under circumstance changes all the time~\cite{evaluating}. This aspect highlights C2 agility's relevance to deal with context changes providing capability of adaptation. In this paper, we address this problem within the scope of simulated environments, which is often addressed in C2~\cite{CC02} in general and in its application areas such as the military domain~\cite{france2014}.

To provide C2 agility, we present a computational model that coordinates and reconfigures entities to handle context changes. Such model is a typed-parameterized extension of a channel system~\citep{modelcheckingBaier}. This extension, hereafter referred to as CS, defines the roles and responsibilities that are executed by the modelled entities. Each member executing roles is modelled as a Dynamic Software Product Line (DSPL)~\citep{Bencomo2008}. To cope with context changes, members can reconfigure themselves or change their coordination structure thereby achieving C2 agility.

Although modern techniques of Model Checking support a large number of states~\citep{modelcheckingBaier}, concurrent systems like ours may suffer from the possibility of states number explosion. \cite{clarkson2014} showed a prototype of Model Checker to QPTL~\citep{QPTL001} resulted from a transformation of HyperCTL formula. However the checker is impractical for real-world programs and does not provide a timely response in runtime when the system is under a dynamic context. Based on this, we evaluate the proposed C2 computational model performing an agility assessment in a software-based simulated environment.

Even the simulation environment being simpler than real scenarios, it is relevant to a myriad of different domains such as the military~\citep{CC03} and environmental monitoring~\cite{simulation001}. Simulation creates different circumstances to evaluate a solution or product, or to train professionals, reducing the need for resources to create real circumstances. Furthermore, many scenarios occur naturally in virtual environment, such as Network Centric Warfare and telemedicine, making simulations relevant to represent events that occur in virtual world
~\citep{telemedicine01, france2014, CC01}.

In our study, we simulate a set of UAVs applied in a reconnaissance mission. The simulation explores different scenarios with context changes. Such changes occur in the mission due to eventually tasks addition, in the entities due to aleatory damages in onboard  entities’ components and parts, or in the environment due to hazard increasing or weather conditions changes, thus causing impacts on the execution. The results of quantitative and qualitative agility metrics indicate that the entities have a certain level of agility. We also identify challenging situations in achieving agility and discuss related tradeoffs. In summary, this work makes the following contributions:

\begin{itemize}
    \item We present a typed-paramaterized channel system modelling C2 system roles, their interactions, and dealing with context changes (Section~\ref{sec:channelSystem});
    \item We design and implement the proposed  channel system and make it  publicly available\footnote{http://github.com/c2} (Section~\ref{sec:designImpl});
    \item We perform a simulation-based study to empirically evaluate the proposed computational model in providing C2 agility, according to quality and quantity metrics (Section~\ref{sec:evaluation}). 
\end{itemize}
