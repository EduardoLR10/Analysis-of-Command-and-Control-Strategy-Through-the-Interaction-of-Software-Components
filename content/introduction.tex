\section{Introduction}
\label{sec:introduction}

% C2 + Example

\textit{Command and Control (C2)} is about focusing the efforts of a set of entities and resources towards the achievement of some task, objective, or goal~\citep{CC02}. The entities may represent individuals, organizations, systems, or a combination of these. Resources involve everything manipulated by the entities, including information exchange. Originally developed in military domain, C2 was based on the idea of a central command concentrating information and power over required elements to accomplish the mission~\citep{CC01}. Moreover, C2 applications include sensitive issues as nuclear weapon and research~\citep{C2-EX2}, national mass-vaccination campaigns~\citep{C2-EX1}, and the COVID-19 pandemic scenario, orchestrating different government and research organizations to find a mitigation or solution to the problem and providing responses to the society~\citep{C2-EX3, C2-EX4, C2-EX5}.

% C2 Agility -> Maneuver + Example + Relevance

\textit{C2 agility} is the entity's capability of dealing with context changes~\citep{france2014}. More complex scenarios caused by new circumstances require entities' collaboration improvements to deal with this dynamism. The Civil Defense domain is an example of scenario where the sudden lack of entities or the increased risk are problems that characterize circumstance changes and requires a timely and suitable C2 application. Hence, C2 Agility needs to handle coordination of the entities, i.e., changing the group's structure and functionalities.  Such ability, so-called  \textit{C2 Maneuver Agility}, combined with the member's configuration, i.e., \textit{C2 Approach Agility}~\citep{alberts2006understanding}, are the two components of the C2 Agility. Moreover, context changes can occur in the mission, environment, or entity.

% Problem

Because a large variety of scenarios has embedded dynamism, it is important to be able to adapt the current organization of entities when is necessary. Otherwise, quality results may drop due to the new circumstances involved. In other words, the lack of C2 Maneuver Agility impacts the performance in these types of missions. Nevertheless, to the best of our knowledge, the current state-of-art does not explore methodologies or strategies to provide C2 Maneuver Agility, especially considering context changes~\cite{france2014, futureC2}.

% Solution

To provide C2 Maneuver Agility, we present a computational model that coordinates entities to handle context changes. Such a model is an extension of the Channel System (CS) presented by~\cite{modelcheckingBaier}. The CS model defines the roles and responsibilities that are executed by the entities. To cope with context changes, members can modify their coordination structure, i.e., information exchanging conditions, to perform the mission, thereby achieving C2 Maneuver Agility.

% Contributions

To asses the proposed computational model, we conduct a simulation. The simulation explores different scenarios with context changes. The results indicate that the entities have higher agility compared to baseline approach. In summary, this work makes the following contributions:

\begin{itemize}
    \item We present a channel system modelling C2 system roles, their interactions, and dealing with context changes (Section~\ref{sec:channelSystem});
    \item We design and implement the proposed channel system and make it  publicly available (Section~\ref{sec:design});
    \item We perform a simulation-based study to empirically estimate the C2 Maneuver Agility impact in scenarios whereas C2 Agility is present (Section~\ref{sec:evaluation}). 
\end{itemize}
