\section{Introduction}
\label{sec:introduction}

% C2 + Example

\textit{Command and Control (C2)} is about focusing the efforts of a set of entities and resources towards the achievement of some task, objective, or goal~\citep{CC02}. The entities may represent individuals, organizations, systems, or a combination of these. Resources involve everything manipulated by the entities, including information exchange. Originally developed in military domain, C2 was based on the idea of a central command concentrating information and power over required elements to accomplish the mission~\citep{CC01}. Moreover, C2 applications include sensitive issues as nuclear weapon and research~\citep{C2-EX2}, national mass-vaccination campaigns~\citep{C2-EX1}, and the COVID-19 pandemic scenario, orchestrating different government and research organizations to find a mitigation or solution to the problem and providing responses to the society~\citep{C2-EX3, C2-EX4, C2-EX5}.

% C2 Agility -> Maneuver + Example + Relevance

\textit{C2 agility} is the entity's capability of dealing with context changes~\citep{france2014}. More complex scenarios caused by new circumstances require changing the collaboration approach among entities to deal with this dynamism in a suitable way. Hence, C2 agility needs to handle reorganization of the entities, i.e, changing the group's structure and functionalities. This branch of C2 agility, \textit{C2 Maneuver Agility}, is relevant due to the complexity of existing scenarios in C2. For instance, the sudden lack of entities or the increased risk of the scenario are some of the problems that characterize circumstance changes of C2 applied in the Civil Defense domain. Such scenario changes characterize context dynamism, which can occur in the mission, environment, or entity.

% Problem

Because a large variety of scenarios has embedded dynamism, it is important to be able to adapt the current organization of entities when is necessary. Otherwise, quality results may drop due to the new circumstances involved. In other words, the lack of C2 Maneuver Agility impacts the performance in these types of missions. Nevertheless, to the best of our knowledge, the current state-of-art does not explore methodologies or strategies to provide C2 Maneuver Agility, especially considering context changes~\cite{france2014, futureC2}.

% Solution

To provide C2 maneuver agility, we present a computational model that coordinates entities to handle context changes. Such model is a typed-parameterized extension of a channel system~\citep{modelcheckingBaier}. This extension, hereafter referred to as CS, defines the roles and responsibilities that are executed by the modelled entities. To cope with context changes, members can communicate and change their coordination structure thereby achieving C2 maneuver agility.

% Contributions

To asses the proposed computational model, we conduct a simulation. The simulation explores different scenarios with context changes. The results indicate that the entities have higher agility compared to baseline approach. We also identify challenging situations in achieving agility and discuss related tradeoffs. In summary, this work makes the following contributions:

\begin{itemize}
    \item We present a typed-paramaterized channel system modelling C2 system roles, their interactions, and dealing with context changes (Section~\ref{sec:channelSystem});
    \item We design and implement the proposed channel system and make it  publicly available\footnote{http://github.com/c2} (Section~\ref{sec:design});
    \item We perform a simulation-based study to empirically evaluate the proposed computational model in providing C2 maneuver agility, according to quality and quantity metrics (Section~\ref{sec:evaluation}). 
\end{itemize}
