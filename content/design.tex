\section{Design Implementation}
\label{sec:design}

Our proposed design, using program graphs and channel systems, models a group of entities coordinating to complete a common goal or objective while managing dynamic events. This approach describes a set of attributes and behaviours to all members and a protocol of communication among them. Thus, these strategies fit as an application of an agent-based model (ABM)~\cite{evaluating}.

An agent-based model allows heterogeneity in the group, due to distinct characteristics of the agents being modeled~\cite{evaluating}, such as sensors onboard or total energy capacity. Also, because agents can be diverse, each agent can react to various situations differently~\cite{evaluating}, according to the information about the environment and an agent’s position within it.

Furthermore, relationships between agents are also defined within an ABM~\cite{evaluating}. The network topology of connections between agents limits their interactions and can have a large impact on the overall system. This topology can be defined statically, where connections do not change over time, or dynamically, where agents adapt who they are connected within a response to the state of the system, e.g, when a maneuver is executed. As agents communicate with each other and sense their surroundings, they can use that information to alter their relationships or positions and self-organize~\cite{evaluating}.

Because ABMs can describe, connected, and heterogeneous agents and capture emergent behaviors~\cite{evaluating}, it is an appropriate abstraction for dynamic scenarios.

% ROLE-BASED MODELING SECTION

Finally, separating a group of reusable functionalities in different roles, i.e, in different program graphs, is an implementation design of role-based modeling~\cite{roleOrientedModeling, modelingAgentOrganizationsUsingRoles}.

Roles are used to form different interfaces for agents in order to restrict the visibility of features~\cite{roleOrientedModeling}, such as internal attributes. Concerning their internal behaviour roles may capture goals and handle responsibilities~\cite{roleOrientedModeling} to execute tasks autonomously. Additionally, as demonstrated in Section \ref{subsec:PG}, roles can be dynamically attached to and retracted from an agent. This feature is especially important if a role shall migrate from one agent to another~\cite{roleOrientedModeling} in run time. For instance, when a maneuver occurs some agents may undergo a complete transformation in its roles, depending on the role of the C2 approach selector to decide which set of roles is more appropriate for each agent.

A role class can be described in terms of its breath and dept~\cite{modelingAgentOrganizationsUsingRoles}. Breadth, or horizontal specialization, addresses the number and complexity of actions supported by a given role~\cite{modelingAgentOrganizationsUsingRoles}. Depth, or vertical specialization, relates to the degree of control an agent can have over its actions and the actions of other agents~\cite{modelingAgentOrganizationsUsingRoles}. Recalling Section \ref{sec:introduction}, horizontal specialization is more related to C2 Approach Agility, due to its nature of providing agility within the same C2 approach, adding resilience and flexibility in the current structure of execution. Similarly, vertical specialization relates to C2 Maneuver Agility. Dept relates to the task allocator and C2 approach selector roles, due to their level of influence over the remaining members while performing a maneuver or even which tasks they will execute.v
