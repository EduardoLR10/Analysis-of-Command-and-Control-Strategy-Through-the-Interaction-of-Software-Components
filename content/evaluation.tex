\section{Evaluation}
\label{sec:evaluation}

To measure the obtained C2 system maneuver agility, we assessed the proposed model under different scenarios based on the definition of the capacity to deal with dynamic circumstances and changing its C2 approach according to some criteria. This assessment is performed through an \textit{in silico} experiment~\cite{simulation01} simulating our motivating example (Section \ref{sec:motivation}). Such a method provides a way to analyze different situations and it allows us to work with different scenarios that would otherwise be unfeasible to test given incurred cost and resources availability. The simulated environment considers that the circumstances can change during the mission execution, allowing to test the effectiveness of the C2 System under such conditions. Thus, our goal with the simulations is to answer the following research question:
 
\begin{center}
\fbox{\begin{minipage}{21em}
\textit{How much C2 maneuver agility does the system have?}
\end{minipage}}    
\end{center}

Next, we define the metrics applied in the evaluation to answer the proposed question and their corresponding descriptions:
\begin{itemize}
    \item Maneuvering (M1): Number of C2 Maneuvers performed by the members to accomplish the mission within a given timeout;
    \item Timeliness (M2): System time, in ticks, to accomplish the mission within a given timeout;
    \item Effectiveness (M3): Percentage of successful tasks completed by the executors.
\end{itemize}

Each scenario of the simulation executed considering two distinct methods of system response, i.e, A1, and A2. With the A1 method, the system starts the execution with an initial context predefined and it performs a task allocation, i.e, distributing the mission received, and in face of context changes, e.g, member dropped, the system keeps running, but it does not perform any kind of adaptation to deal with new circumstances. In its turn, the A2 method applies the C2 computational design proposed in Section~\ref{sec:channelSystem} and whose implementation is described in Section~\ref{sec:design}. Moreover, because this simulation was a collaborative project, i.e, evaluate not just C2 maneuver agility, the A2 method also assesses the performance considering members that are capable of enabling and disabling sensors throughout the execution. Ultimately, we compare the results obtained from each method, i.e, answer for the research question, to confirm the presence of an increased level of C2 maneuver agility.

%%%%%%%%%%%%%%%%%%%%%%%%%%%%%%%%%%
% SUBSECTION 
%%%%%%%%%%%%%%%%%%%%%%%%%%%%%%%%%%
\subsection{Simulation Scenarios}
\label{sub:scenarios}

A simulation scenario is composed of an initial context and a sequence of events that provide dynamism. The initial context comprises the set $E$ of members operating an initial C2 Approach $\omega$, the mission $M$ composed of a set of tasks, and the environment. Each possible event, e.g, \textit{memberFailure}, \textit{sensorFailure}, and \textit{envChange}, represents an action that causes a member or environment changes in runtime. These actions are called during simulation to create a dynamic scenario, resembling realistic settings.

The environment represents all the conditions of the place where the members act, e.g, weather conditions, hazard, and communication restrictions, and it is modeled as the state of a specific kind of onboard sensor, e.g, a foggy day can turn a VGA sensor useless.

The scenarios were created with different sequences of actions (Table \ref{tab:scenarios}), i.e, context changes, combined with the same initial context, i.e., members, mission, environment, and the initial C2 Approach. During the simulation, all these changes occur within the timeout, i.e., mission time limit, but not equally distributed over time. The presented sequence is attended but the exact moment is aleatory.

\begin{table}[h]
\centenring
\fontsize{9}{9}
\selectfont
\caption{List of events(\textit{EC-envChange; SF-sensorFailure; MF-memberFailure}) that characterizes the context changes within the scenarios tested. The initial C2 Approach, the set of members $E$ and the mission $M$ remain unchanged.}
\label{tab:scenarios}
\begin{tabular}{|m{0.1\textwidth}|m{0.84\textwidth}|}
\hline
\rowcolor{lightgray}
 \textbf {Scenario} & \hfil  \textbf {Context Changes} \\
\hline
 \hfil 1 & EC $\rightarrow$ EC $\rightarrow$ EC $\rightarrow$ EC $\rightarrow$ EC\\
\hline 
 \hfil 2 & EC $\rightarrow$ EC $\rightarrow$ EC $\rightarrow$ EC $\rightarrow$ EC $\rightarrow$ EC $\rightarrow$ EC $\rightarrow$ EC $\rightarrow$ EC $\rightarrow$ EC $\rightarrow$ EC $\rightarrow$ EC $\rightarrow$ EC $\rightarrow$ EC $\rightarrow$ EC \\
\hline 
 \hfil 3 & 
EC $\rightarrow$ EC $\rightarrow$ EC $\rightarrow$ EC $\rightarrow$ EC $\rightarrow$ MF $\rightarrow$ EC $\rightarrow$ EC $\rightarrow$ EC $\rightarrow$ EC $\rightarrow$ EC $\rightarrow$ MF $\rightarrow$ EC $\rightarrow$ EC $\rightarrow$ EC $\rightarrow$ EC $\rightarrow$ EC $\rightarrow$ MF $\rightarrow$ EC $\rightarrow$ EC $\rightarrow$ EC $\rightarrow$ EC $\rightarrow$ EC  \\
\hline 
 \hfil 4 & EC $\rightarrow$ EC $\rightarrow$ EC $\rightarrow$ EC $\rightarrow$ EC $\rightarrow$ SF $\rightarrow$ EC $\rightarrow$ EC $\rightarrow$ EC $\rightarrow$ EC $\rightarrow$ EC $\rightarrow$ SF $\rightarrow$ EC $\rightarrow$ EC $\rightarrow$ EC $\rightarrow$ EC $\rightarrow$ EC $\rightarrow$ SF $\rightarrow$ EC $\rightarrow$ EC $\rightarrow$ EC $\rightarrow$ EC $\rightarrow$ EC  \\
\hline 
 \hfil 5 & EC $\rightarrow$ EC $\rightarrow$ EC $\rightarrow$ SF $\rightarrow$ EC $\rightarrow$ EC $\rightarrow$ EC $\rightarrow$ MF $\rightarrow$ EC $\rightarrow$ EC $\rightarrow$ EC $\rightarrow$ SF $\rightarrow$ EC $\rightarrow$ EC $\rightarrow$ EC $\rightarrow$ MF $\rightarrow$ EC $\rightarrow$ EC $\rightarrow$ EC $\rightarrow$ SF $\rightarrow$ EC $\rightarrow$ EC $\rightarrow$ EC $\rightarrow$ MF  \\
\hline 
 \hfil 6 & EC $\rightarrow$ SF $\rightarrow$ MF $\rightarrow$ EC $\rightarrow$ SF $\rightarrow$ EC $\rightarrow$ SF $\rightarrow$ EC $\rightarrow$ SF $\rightarrow$ MF $\rightarrow$ EC $\rightarrow$ SF $\rightarrow$ EC $\rightarrow$ SF $\rightarrow$ EC $\rightarrow$ SF \\
\hline 
 \hfil 7 & MF $\rightarrow$ SF $\rightarrow$ EC $\rightarrow$ EC $\rightarrow$ EC $\rightarrow$ EC $\rightarrow$ EC $\rightarrow$ EC $\rightarrow$ EC $\rightarrow$ EC $\rightarrow$ EC $\rightarrow$ EC $\rightarrow$ MF $\rightarrow$ SF $\rightarrow$ EC $\rightarrow$ EC $\rightarrow$ EC $\rightarrow$ EC $\rightarrow$ EC $\rightarrow$ MF $\rightarrow$ SF $\rightarrow$ EC $\rightarrow$ EC $\rightarrow$ EC $\rightarrow$ MF $\rightarrow$ SF $\rightarrow$ EC \\
 \hline
 \hfil 8 & MF $\rightarrow$ SF $\rightarrow$ EC $\rightarrow$ MF $\rightarrow$ SF $\rightarrow$ EC $\rightarrow$ MF $\rightarrow$ SF $\rightarrow$ EC $\rightarrow$ MF $\rightarrow$ SF $\rightarrow$ EC \\
 \hline
 \hfil 9 & SF $\rightarrow$ SF $\rightarrow$ SF $\rightarrow$ MF $\rightarrow$ SF $\rightarrow$ SF $\rightarrow$ SF $\rightarrow$ MF $\rightarrow$ SF $\rightarrow$ SF $\rightarrow$ SF $\rightarrow$ MF $\rightarrow$ SF $\rightarrow$ SF $\rightarrow$ SF \\
 \hline
 \hfil 10 & SF $\rightarrow$ EC $\rightarrow$ MF $\rightarrow$ SF $\rightarrow$ EC $\rightarrow$ MF $\rightarrow$ SF $\rightarrow$ EC $\rightarrow$ MF $\rightarrow$ SF $\rightarrow$ EC $\rightarrow$ MF $\rightarrow$ EC $\rightarrow$ EC $\rightarrow$ EC $\rightarrow$ EC $\rightarrow$ EC $\rightarrow$ EC $\rightarrow$ EC $\rightarrow$ EC $\rightarrow$ EC $\rightarrow$ EC \\
 \hline
\end{tabular}

\end{table}


The simulation operates a scenario with 5 possible types of tasks (0 to 4) and 5 types of sensors (A, B, C, D, and E). The tasks and sensors onboard are randomly chosen before the running round. When the tasks allocation process starts, the algorithm applies a function that returns the quality $Q_{ij}$, obtained from a table, that correlates a sensor $i$ to the task $j$. When $Q_{ij}=0$ it means the sensor $i$ is not able to perform the task $j$.

A context change simulated by the system causes the quality reduction of a specific type of sensor, e.g., an environment change simulating a luminosity decreasing reduces in $50\%$ the quality of the sensor type 2 that represents a VGA camera. In case the sensor burns out, its quality comes to be zero and the task will be transmitted to another member to check its execution capacity. Also, the member with a burned sensor is capable of enabling and disabling sensors depending on the task it is currently executing. For instance, after the quality reduction of its sensor type 2, it disables the defective sensor while enabling the best sensor for its next target, present in its allocation list. Furthermore, we can lose a member and, consequently, all its sensors onboard.

Context changes can generate situations such as the tasks reallocation is not enough to keep mission execution. In that case, the C2 System performs a C2 Approach change, i.e., maneuvering. The new C2 Approach operated can provide a higher awareness level, i.e., more information shared by the members with a new communication structure, and it can help the members perform reallocation based on the new data exchanged. The initial C2 Approach for all scenarios simulated is De-Conflicted, i.e, a ring communication structure. From this C2 Approach, the maneuver follows incrementally over a specific order of C2 approaches, based on the following research~\cite{france2014}, and going back to Conflicted after Edge. The Conflicted is the final C2 Approach adopted by the system before discarding the unfeasible tasks.

%%%%%%%%%%%%%%%%%%%%%%%%%%%%%%%%%%
% SUBSECTION 
%%%%%%%%%%%%%%%%%%%%%%%%%%%%%%%%%%
\subsection{Experimental Setup}
\label{sub:setup}

Figure \ref{fig:exp_setup} shows the experimental setup applied to assess C2 maneuver agility. A factorial experiment is performed with the treatments applied to the simulation scenario (Section~\ref{sub:scenarios}) and the methods. Combinations of the factors are assessed with the metrics described in Section~\ref{ssec:definition}.

\begin{figure}[ht!]
    \centering
    \scalebox{.65}{\input{tikz/experimentalDesign}}
    \caption{Experimental setup with the treatments performed by the simulator}
    \label{fig:exp_setup}
\end{figure}

The simulation uses a set of random variables to define some elements during execution, i.e, UAVs, and tasks position, and sensor that will be affected by the context event. For a consistent comparison between the methods A1 and A2, the same set of random variables is used by both methods during the same execution number.

Based on the following research~\cite{CochranW.G.1983}, the factorial experiment with the scenario and action method factors (Figure~\ref{fig:exp_setup}) was executed 500 times for all combinations of treatments. This sample size satisfies $95\%$ of confidence level~\cite{CochranW.G.1983}.

All scenarios created from the variables' values definition consider the same size of members' set $|E|=5$, the mission size $|M|=30$, and an initial C2 Approach $\omega_0 =$ De-Conflicted (Figure \ref{fig:scenario}). Finally, each simulation scenario has a deadline of 1000 ticks of execution time.

%%%%%%%%%%%%%%%%%%%%%%%%%%%%%%%%%%
% SUBSECTION 
%%%%%%%%%%%%%%%%%%%%%%%%%%%%%%%%%%
\subsection{Results and Analysis}

Table~\ref{table:results02} and Figures~\ref{fig:m1},~\ref{fig:m2}, and~\ref{fig:m3} show the results obtained by the simulation. Figure~\ref{fig:m1} shows the maneuvering (M1) metric results for all scenarios, not present in the A1 method, with a more constant value due to the cost to perform a C2 Approach change. This result points out a huge difference in C2 maneuver agility levels between the two methods. Moreover, even though sensor reconfigurations are happening in all scenarios, the ones with greater maneuvers values (M1) pointed out better results (M3) when compared with the A1 method.

\begin{figure}[ht]
\centering
\begin{minipage}{.5\textwidth}
    \centering
    \small
    \fontsize{7}{7}\selectfont
	\captionof{table}{Metrics results}
    \label{table:results02}
    \begin{tabular}{rrrr} \hline
		& \bf{M1}
        & \bf{M2}
		& \bf{M3} \\  \hline 
		
		& Mean(St.Dev.)  & Mean(St.Dev.) & Mean(St.Dev.)\\ [1ex]
		
		\multicolumn{3}{l}{\textbf{$\longrightarrow$ Scenario 1 }} \\
	% Scenario  1 
\bf{A1}  & 0.0 ($\pm$0.0)  & 890.0 ($\pm$66.8)  & 82.0 ($\pm$5.9)  \\
\bf{A2}  & 1.7 ($\pm$0.3)  & 945.5 ($\pm$33.1)  & 97.7 ($\pm$2.9)  \\ [1ex]
	
	\multicolumn{3}{l}{\textbf{$\longrightarrow$ Scenario 2 }} \\
% Scenario  2 
\bf{A1}  & 0.0 ($\pm$0.0)  & 866.5 ($\pm$72.3)  & 77.5 ($\pm$6.3) \\
\bf{A2}  & 2.6 ($\pm$0.6)  & 934.0 ($\pm$33.0)  & 95.2 ($\pm$4.2) \\ [1ex]
	
	\multicolumn{3}{l}{\textbf{$\longrightarrow$ Scenario 3 }} \\
% Scenario  3 
\bf{A1}  & 0.0 ($\pm$0.0)  & 782.1 ($\pm$78.3)  & 63.7 ($\pm$5.5) \\
\bf{A2}  & 2.8 ($\pm$0.3)  & 883.0 ($\pm$49.2)  & 69.1 ($\pm$4.5) \\ [1ex]
	
	\multicolumn{3}{l}{\textbf{$\longrightarrow$ Scenario 4 }} \\
% Scenario  4 
\bf{A1}  & 0.0 ($\pm$0.0)  & 799.1 ($\pm$77.0)  & 63.6 ($\pm$5.3) \\
\bf{A2}  & 3.0 ($\pm$0.1)  & 913.4 ($\pm$35.9)  & 84.5 ($\pm$5.1) \\ [1ex]
	
	\multicolumn{3}{l}{\textbf{$\longrightarrow$ Scenario 5 }} \\
% Scenario  5 
\bf{A1}  & 0.0 ($\pm$0.0)  & 733.7 ($\pm$75.5)  & 54.6 ($\pm$4.6)  \\
\bf{A2}  & 2.9 ($\pm$0.2)  & 887.7 ($\pm$48.4)  & 70.7 ($\pm$5.1) \\ [1ex]
	
	\multicolumn{3}{l}{\textbf{$\longrightarrow$ Scenario 6 }} \\
% Scenario  6 
\bf{A1}  & 0.0 ($\pm$0.0)  & 700.6 ($\pm$79.4)  & 42.0 ($\pm$4.5) \\
\bf{A2}  & 2.9 ($\pm$0.2)  & 910.5 ($\pm$59.3)  & 65.7 ($\pm$5.9) \\ [1ex]
	
	\multicolumn{3}{l}{\textbf{$\longrightarrow$ Scenario 7 }} \\
% Scenario  7 
\bf{A1}  & 0.0 ($\pm$0.0)  & 686.2 ($\pm$58.4)  & 41.1 ($\pm$3.5)  \\
\bf{A2}  & 2.8 ($\pm$0.3)  & 820.5 ($\pm$61.4)  & 60.1 ($\pm$5.6)  \\ [1ex]
	
	\multicolumn{3}{l}{\textbf{$\longrightarrow$ Scenario 8 }} \\
% Scenario  8 
\bf{A1}  & 0.0 ($\pm$0.0)  & 629.7 ($\pm$68.6)  & 36.3 ($\pm$3.8) \\
\bf{A2}  & 2.8 ($\pm$0.3)  & 785.1 ($\pm$58.3)  & 52.1 ($\pm$4.7) \\ [1ex]
	
	\multicolumn{3}{l}{\textbf{$\longrightarrow$ Scenario 9 }} \\
% Scenario  9 
\bf{A1}  & 0.0 ($\pm$0.0)  & 491.0 ($\pm$74.2)  & 26.4 ($\pm$3.7) \\
\bf{A2}  & 3.0 ($\pm$0.1)  & 743.2 ($\pm$66.1)  & 53.2 ($\pm$5.1) \\ [1ex]
	
	\multicolumn{3}{l}{\textbf{$\longrightarrow$ Scenario 10 }} \\
		
% Scenario  10 
\bf{A1}  & 0.0 ($\pm$0.0)  & 393.4 ($\pm$81.2)  & 24.3 ($\pm$2.6) \\
\bf{A2}  & 2.9 ($\pm$0.2)  & 650.7 ($\pm$114.8)  & 38.8 ($\pm$4.7) \\ [1ex]
		\hline
	\end{tabular}
\end{minipage}%
\begin{minipage}{.5\textwidth}
  \centering
  \includegraphics[width=0.95\linewidth]{figures/graphs/Boxplot_M2.png}
  \captionof{figure}{Total of maneuverings (M1)}
  \label{fig:m1}
\end{minipage}
\end{figure}

\begin{figure}
\centering
\begin{minipage}{.5\textwidth}
  \centering
  \includegraphics[width=0.95\linewidth]{figures/graphs/Boxplot_M3.png}
  \captionof{figure}{Total operation time (M2)}
  \label{fig:m2}
\end{minipage}%
\begin{minipage}{.5\textwidth}
  \centering
  \includegraphics[width=0.95\linewidth]{figures/graphs/Boxplot_M5.png}
  \captionof{figure}{Completed tasks (M3)}
  \label{fig:m3}
\end{minipage}
\end{figure}

Notice in Figure~\ref{fig:m2} a significant difference between the system total time in the operation of the action methods, i.e, timeliness (M2). The choice of C2 Approach and how the C2 Approach space is explored, aspect only present in A2, come up against limitations of the entities' communication structure, as well as the cost for performing such maneuvers, i.e., time spent and entities' autonomy reduction. Although it took longer to complete the mission, the A2 method allowed the system to keep running for a longer time even with more context events happening due to the added execution time compared with the A1 method. More time consumed indicates a capacity to keep running under context perturbations and more resilience of the system to maintain execution even under such context changes. 

Figure~\ref{fig:m3} shows that the capacity of the C2 system to solve tasks, under the effect of changes in the context that alter its functioning, is more significant in the A2 method. This result can be explained by the reconfiguration of sensors in each member combined with the reorganization of the entire team via a maneuver. The static nature of the A1 method limits its performance since the initial configuration is the only and unique option to execute the mission.

With a Shapiro-Wilk test~\citep{stat001} a \textit{p-value} less than 0.05 was obtained, thus indicating a non-normal distribution. Based on this, an Mann-Whitney U Test described in~\cite{stat002} was applied, i.e, Wilcoxon-test in R, to check differences among the samples of each action method applied, i.e, A1, and A2. According to the number and value of samples, the statistical analysis confirmed the difference between all results from A1 and A2 methods, collected in all scenarios tested.

In summary, the empirical findings indicate a better overall performance using the A2 method, despite its longer total execution time. Furthermore, it is possible to observe that, in the majority of the scenarios, the gain in the total amount of completed tasks is greater than the extra required time to finish the mission with the proposed A2 method. Such a method enables the system to keep more time in action and allow a greater number of completed tasks, by enabling and disabling sensors and executing maneuvers. The capabilities embedded in the system result in a resilience increasing and enable the behavioral model the ability to cope with context changes and perturbations in dynamic scenarios.