\section{Conclusion}
\label{sec:conclusion}

Using the evaluation data (Section \ref{sec:evaluation}) as our base, we can conclude that providing more options to the agents, with different C2 approaches through maneuvers, increased performance in execution. This capability brings resilience and robustness to the overall system, due to the treatment of unexpected events by changing patterns of interactions, decision rights, and distribution of information, i.e, changing the team's organization.

The proposed design (Section \ref{sec:design}) has covered a lot of nuances in the simulation, e.g, sensor failure, drop of members. Sudden changes along execution were perceptible to the program graphs and were properly communicated to the right members using the channel systems, avoiding a significant drop in quality of the mission. Further, the ABM's roles concept allowed the system to change its configuration, necessary to perform maneuvers.

Furthermore, as future work, an evolution of the simulation would be to add more types of dynamic events, such as changing the mission, i.e, adding or removing tasks during execution, to measure the performance of the proposed design with a wider range of context changes. Moreover, because of the implementation's modularity nature, other functions can be added to allocate tasks or detect the need for maneuvering depending on some criteria, e.g, if energy and duration are not a concern, artificial intelligence and big data can be used to allocate tasks. Also, different metrics can be used to evaluate the model, such as resilience (\textit{How can we maintain our current level of performance?}) and quality of tasks execution (\textit{How to execute tasks balancing best sensor to the task and responsiveness?}). Finally, establishing more precisely how each type of agility, i.e., C2 Approach Agility and C2 Maneuver Agility, impacts C2 Agility is also a relevant contribution to this work.

In conclusion, evidence provided by the simulation indicates that the presented design seems adequate to specify problems in the Command and Control domain. The model seems to provide enough awareness to detect context changes and it equips the system with options of how to deal with them by increasing the C2 maneuver agility level.