\section{Conclusion}
\label{sec:conclusion}

Using the evaluation data (Section \ref{sec:evaluation}) as our base, we can conclude that providing more options to the agents, with different C2 approaches through maneuvers, increased performance in execution. This capability brings resilience and robustness to the overall system, due to the treatment of unexpected events by changing patterns of interactions, decision rights and distribution of information, i.e, changing the teams organization.

Ultimately, the proposed design (Section \ref{sec:design}) has covered a lot of nuances in the simulation, e.g, sensor failure, drop of members. Sudden changes along execution were perceptible to the program graphs and were properly communicated to the right members using the channel systems, avoiding a significant drop in quality of the mission. In conclusion, evidence provided by the simulation indicates that the presented design seems adequate to specify problems in the Command and Control domain, considering context changes and how to deal with them by increasing the systems C2 maneuver agility level.