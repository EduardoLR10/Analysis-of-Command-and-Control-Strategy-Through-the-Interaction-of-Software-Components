\section{Conclusion}
\label{sec:conclusion}

To address the lack of a precise and uniform description of product-line analyses and their properties, we propose a framework consisting of a machine-verified theory comprising formal specification and verification of key concepts and properties of product-line analyses. In particular, it defines abstract functions and types modeling essential abstractions in this problem domain, such as analysis steps, models, and intermediate analysis results. Product-line analyses are %then presented 
modeled in a compositional manner, providing an overall understanding of the structure of the space of family-based, feature-based, and product-based 
analyses, showing how the different types of product-line analyses compose and inter-relate. Additionally, we provide mechanized proofs of commutativity of different analysis strategies. Therefore, the novelty of this work lies in how we model and map the different strategies and how we prove certain properties. 

To create the %supporting 
analysis framework, we reviewed a number of existing analyses for the different models and properties and identified essential abstractions of such analyses and their structure.
\Cref{fig:strategies-generic} depicts the patterns that were found during our
research, relating annotative and compositional models as well as the
operations defined over them.
Such view allows the organization and structuring of facts (e.g.,
commutativity of intermediate analysis steps) in a concise and precise
manner, facilitating the communication of ideas and contributing to a more
comprehensive understanding of underlying principles used in these
strategies.
Indeed, we were able to fit different types of analyses to our framework, in terms of
the employed techniques as well as the models and properties under analysis.

Moreover, the commuting diagram in \Cref{fig:strategies-generic} and the
corresponding mechanized commutativity theory may be leveraged as guidelines to the
formalization of existing analysis strategies and to the design of new ones.
Hence, this work contributes to the ongoing search for a
principled and possibly automated way to lift a given specification and
analysis technique to product lines~\cite{Thum2014}.

% Future work
As next steps, we aim at formalizing the lower quadrants of the commuting diagram (\Cref{fig:strategies-generic}) along the guidelines discussed in Section~\ref{sec:abstraction-description}.
We also plan to extend the qualitative assessment conducted in \Cref{sec:frameworkInstances} to encompass more analyses as well as to formally instantiate existing and new analyses.
An additional goal is to model product-line evolution within our
analysis framework, so that one is able to reuse analysis effort throughout the
lifetime of products.
Finally, we also intend to provide a reference implementation of the framework and tool support for its derivation process.